\chapter{Conclusiones\label{sec:conclusiones}}

Para la realización de este proyecto ha sido necesario aplicar conocimientos relacionados con casi todas las ramas de las telecomunicaciones.

La rama \textbf{electrónica} es claramente la \textbf{dominante} durante la realización del proyecto, pues el diseño del esquemático, la \acrshort{PCB}, y el montaje de todo el sistema sería imposible sin los conocimientos adquiridos durante mi estancia en la Universidad. Aun así, es posible apreciar ciertos \textbf{matices relacionados con el resto de las ramas}. Gestión de redes, protocolos de comunicación y el filtrado de las señales han sido utilizados para conseguir los resultados ya mostrados.

El diseño de la \acrshort{PCB} fue muy interesante ya que hasta este momento todos los diseños realizados habían sido orientados al análisis teórico y simulaciones, dejando de lado la \textbf{implementación final de un sistema} de esas características.

Una de las partes del proyecto que \textbf{más trabajo} ha supuesto y que menos se ve reflejado en esta memoria o en el resultado final es el \textbf{desarrollo del firmware del microcontrolador STM32F4}. Este microcontrolador tiene una gran comunidad que lo respalda y un buen número de referencias, pero carece de un sistema de aprendizaje autónomo o guías de iniciación. Esto provoca que el \textbf{comienzo} del desarrollo resulte especialmente \textbf{complicado}.

El sistema ha cumplido ampliamente con las expectativas. Presenta además \textbf{mejores características} y un \textbf{presupuesto muy inferior} al de su competidor directo, OpenBCI. El coste final del proyecto es de 113€\footnote{El presupuesto se explica con más detalle en el apéndice \ref{sec:BOM}} contando la tarjeta de adquisición y trabajando con 16 canales, mientras que si se desea utilizar OpenBCI con WiFi y es necesario hacer un desembolso de 300€ y sólo puede trabajar con 4 canales, 8 en el mejor de los casos.

\section{Futuras mejoras}

Por supuesto, aunque el proyecto final ha sido un éxito, no se debe olvidar que este sistema es un \textbf{prototipo inicial} y que no se había trabajado antes con ciertos elementos que juegan un papel principal de modo que es natural que presente ciertos puntos que podrían ser mejorables en el futuro.

En futuras iteraciones sobre este proyecto sería deseable \textbf{ampliar las características} ya presentes dotando al sistema de ciertas funciones que no se han implementado por falta de tiempo y recursos.\\
Algunas de estas características son la adquisición en tiempo real, la parametrización del número de muestras y los tipos de filtros y la inclusión en el software de un sistema de almacenamiento integrado (ya contemplado en el hardware). Igualmente, añadir el \textbf{sistema de \textit{debug} por \textsc{\acrshort{jtag}}} facilitaría el desarrollo del firmware del microcontrolador.

La mejora más importante podría ser el desarrollo de una \acrshort{PCB} en la que se encuentren \textbf{todos los elementos integrados}, aumentando la comodidad al trabajar con el sistema y su portabilidad.


