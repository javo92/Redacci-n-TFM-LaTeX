\chapter{Estado del Arte\label{sec:EstadoDelArte}}

Con el paso del tiempo se ha demostrado que una de nuestras mayores virtudes como seres humanos es la habilidad de aprovechar el saber cultivado por otras personas para realizar nuevos descubrimientos con mayor facilidad. En la actualidad con la ayuda de Internet esta ventaja se ha visto potenciada hasta límites insospechados.

Como se ha mencionado con anterioridad en el capítulo \ref{sec:introduccion}, a lo largo de los años se han desarrollado numerosas alternativas a los dispositivos presentes en los hospitales y laboratorios utilizados normalmente para el estudio del cerebro. 
\\Aunque se han invertido muchos recursos en estos dispositivos, el objetivo es permitir ampliar el número de personas capaces de estudiar el cerebro humano, consiguiendo  así aumentar las posibilidades de mejorar nuestro conocimiento sobre el mismo.

De esta forma debería ser más fácil realizar nuevos descubrimientos como, por ejemplo, encontrar nuevas formas de diagnosticar enfermedades o de realizar una comunicación hombre-máquina para aquellas personas que por un motivo u otro no pueden utilizar los medios convencionales.

A lo largo de este capítulo se presentarán algunos de los dispositivos capaces de capturar un EEG haciendo uso de electrodos, algunos a nivel personal, otros enfocados a la docencia y, por supuesto, algunos diseñados por empresas con el fin de realizar un producto final.


Historia procesadores (Gráfica de procesadores)
Antecedentes de la placa (Open BCI)