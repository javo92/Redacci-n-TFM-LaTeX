
% Para que compile hace falta tener instalado Perl y ejecutar
% makeglosaries.exe main desde powershell/cmd

% Acrónimos

%TODO: Añadir aquí los acrónimos
% Ejemplo de acrónimo
%\newacronym{FPGA}{FPGA}{Field-Programmable Gate Array}
\newacronym{PCB}{PCB}{Print Board Circuit}
\newacronym{EEG}{EEG}{Electroencefalograma}
\newacronym{ECG}{ECG}{Electrocardiograma}
\newacronym{ADC}{ADC}{Convertidores Analógico-Digital}
\newacronym{SPI}{SPI}{Serial Peripheral Interface}
\newacronym{GPIO}{GPIO}{General Purpose Input/Output}
\newacronym{USB}{USB}{Universal Serial Bus}
\newacronym{CPU}{CPU}{central processing unit}
\newacronym{RAM}{RAM}{Random Access Memory}
\newacronym{SoC}{SoC}{System on Chip}
\newacronym{UART}{UART}{Universal Asynchronous Receiver-Transmitter}
\newacronym{IDE}{IDE}{Integrated Development Environment}
\newacronym{SPS}{SPS}{Samples Per Second}
\newacronym{MCU}{MCU}{Microcontroller Unit}
\newacronym{DSP}{DSP}{Digital Signal Processing}
\newacronym{FCPU}{FCPU}{Frecuencia de la CPU}
\newacronym{FPU}{FPU}{Floating Point Unit}
\newacronym{LQFP64}{LQFP64}{ 64 pin Low-profile Quad Flat Package}
\newacronym{OTG}{OTG}{On The Go}
\newacronym{LED}{LED}{Light-Emitting Diode}
\newacronym{PLL}{PLL}{Phase-Locked Loop}
\newacronym{SMT}{SMT}{Surface-Mount Technology}
\newacronym{THT}{THT}{Through-Hole Technology}
\newacronym{HSE}{HSE}{High-Speed External clock}
\newacronym{LSI}{LSI}{Low-Speed Internal clock}
\newacronym{CSS}{CSS}{Clock Security System}
\newacronym{HAL}{HAL}{Hardware Abstraction Layer}
\newacronym{FIR}{FIR}{Finite Impulse Response}
\newacronym{SNR}{SNR}{Signal-to-Noise Ratio}
\newacronym{jtag}{jtag}{Joint Test Action Group}
% Glosario

%TODO: Añadir aquí las definiciones del glosario
% Ejemplo de glosario
\newglossaryentry{bitstream}{
	name={bitstream},
	description={En este contexto se refiere al binario que configura el Hardware de la FPGA}}

\newglossaryentry{Full Duplex}{
	name={Full Duplex},
	description={Cualidad de los elementos que permiten la entrada y salida de datos de forma simultánea}}

\newglossaryentry{Tension de Dropout}{
	name={Tensión de Dropout},
	description={Mínima diferencia de tensión entre la entrada y la salida dentro de la cual el circuito es todavía capaz de regular la salida dentro de las especificaciones}}
	
\newglossaryentry{Open Source}{
	name={Open Source},
	description={Filosofía aplicada a la realización de un proyecto (Software o Hardware) que implica una colaboración abierta, proporcionando aquella documentación necesaria para replicar dicho proyecto y proporcionando la libertad de aprovechar y/o mejorarlo sin restricciones}}
