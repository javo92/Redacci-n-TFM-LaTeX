\chapter{Estado del Arte\label{sec:EstadoDelArte}}

Con el paso del tiempo se ha demostrado que una de nuestras mayores virtudes como seres humanos es la habilidad de aprovechar el saber cultivado por otras personas para realizar nuevos descubrimientos con mayor facilidad. En la actualidad con la ayuda de Internet esta ventaja se ha visto potenciada hasta límites insospechados.

Como se ha mencionado con anterioridad en el capítulo \ref{sec:introduccion}, a lo largo de los años se han desarrollado numerosas alternativas a los dispositivos presentes en los hospitales y laboratorios utilizados normalmente para el estudio del cerebro. 
\\Aunque se han invertido muchos recursos en estos dispositivos, el objetivo es permitir ampliar el número de personas capaces de estudiar el cerebro humano, consiguiendo  así aumentar las posibilidades de mejorar nuestro conocimiento sobre el mismo.

De esta forma debería ser más fácil realizar nuevos descubrimientos como, por ejemplo, encontrar nuevas formas de diagnosticar enfermedades o de realizar una comunicación hombre-máquina para aquellas personas que por un motivo u otro no pueden utilizar los medios convencionales.

A lo largo de este capítulo se presentarán algunos de los dispositivos capaces de capturar un EEG haciendo uso de electrodos, algunos a nivel personal, otros enfocados a la docencia y, por supuesto, algunos diseñados por empresas con el fin de realizar un producto final que vender a terceros.
\\Aunque éstos dispositivos pueden presentar características muy diversas en función de la persona que los crea y el objetivo del mismo, normalmente se pueden dividir en dos grandes grupos. Dependiendo del tipo de electrodo que se utilice para captar las señales se hablará de dispositivos basados en electrodos húmedos o en electrodos secos.

\section{Tipos de electrodos\label{sec:Tipos_Electrodos}}

Los electrodos hacen la función de interfaz adaptadora entre los distintos medios por los que se transmiten las señales.

Rellenar hablando de como son los electrodos, cómo se suelen usar, alguna imagen donde se vea un electrodo y algún esquema eléctrico.

Comparar los electrodos con la función de los huesos del oído al adaptar la señal acústica para que llegue mejor al oído.

\subsection{Electrodos húmedos\label{sec:Elec_humedos}}

Los electrodos húmedos, 

IMAGEN de ejemplo de electrodos húmedos.

Ventajas e inconvenientes.

\subsection{Electrodos secos\label{sec:Elec_secos}}

IMAGEN de ejemplo de electrodos húmedos.

Ventajas e inconvenientes.

\section{Dispositivos similares\label{sec:Disp_similares}}



\subsection{Proyectos personales\label{sec:Pro_personales}}

En internet se pueden encontrar algunos ejemplos de personas que han dedicado su tiempo a crear dispositivos capaces de captar un EEG...

Hablar de los códigos accesibles en distintas páginas web, github y otros sitios.

Explicar que me he basado en ellos para agilizar el diseño ya que las definiciones de los distintos registros son similares.

\subsection{Proyectos docentes\label{sec:Pro_docentes}}

Hablar del de Nerea, como se ha trabajado con el y en que se basa.

\subsection{Comerciales\label{sec:Pro_empresa}}

Ejemplos comerciales como zanna u otras empresas.

Sistemas BCI comerciales.

Claramente todos los ejemplos anteriores presentan un coste bastante dispar entre ellos costando desde los cientos de euros de los proyectos personales y docentes hasta algunos miles de euros en el caso de productos comerciales.

Historia procesadores (Gráfica de procesadores)
Antecedentes de la placa (Open BCI)