% Definiciones y constantes de estilo
% Clase del documento
\documentclass[a4paper,12pt,twoside,openright,titlepage]{book}

%
% Paquetes necesarios
%

% Símbolo del euro
\usepackage{eurosym}
% Codificación UTF8
\usepackage[utf8]{inputenc}
% Caracteres del español
\usepackage[spanish]{babel}
% Código, algoritmos, etc.
%\usepackage{listings}
\usepackage{listingsutf8}
% Definición de colores
\usepackage{color}
% Extensión del paquete color
\usepackage[table,xcdraw]{xcolor}
% Márgenes
\usepackage{anysize}
% Cabecera y pie de página
\usepackage{fancyhdr}
% Estilo título capítulos
\usepackage{quotchap}
% Algoritmos (expresarlos mejor)
\usepackage{algorithmic}
% Títulos de secciones
\usepackage{titlesec}
% Fórmulas matemáticas
\usepackage[cmex10]{amsmath}
% Enumeraciones
\usepackage{enumerate}
% Páginas en blanco
\usepackage{emptypage}
% Separación entre cajas
\usepackage{float}
% Imágenes
\usepackage[pdftex]{graphicx}
% Mejora de las tablas
\usepackage{array}
% Mejora de los símbolos matemáticos
\usepackage{mdwmath}
% Separar figuras en subfiguras
%\usepackage[caption=false,font=footnotesize]{subfig}
% Incluir pdfs externos
\usepackage{pdfpages}
% Mejoras sobre las cajas
\usepackage{fancybox}
% Apéndices
\usepackage{appendix}
% Marcadores (para el pdf)
\usepackage{bookmark}
% Estilo de enumeraciones
\usepackage{enumitem}
% Espacio entre líneas y párrafos
\usepackage{setspace}
% Glosario/Acrónimos
%\usepackage{hyperref}
\usepackage[acronym,automake]{glossaries}
\makeglossaries
% Fuentes
\usepackage[T1]{fontenc}
%\setmainfont{Times New Roman}
% Bibliografía
\usepackage[sorting=none,natbib=true,backend=bibtex,bibencoding=ascii]{biblatex}
% Fix biblatex+babel warning
\usepackage{csquotes}

% Enlaces
\hypersetup{hidelinks,pageanchor=true,colorlinks,citecolor=Fuchsia,urlcolor=black,linkcolor=Cerulean}

% Euro (€)
\DeclareUnicodeCharacter{20AC}{\euro}

% Inclusión de gráficos
\graphicspath{{./graphics/}}

% Texto referencias
\addto{\captionsspanish}{\renewcommand{\bibname}{Bibliografía}}

% Extensiones de gráficos
\DeclareGraphicsExtensions{.pdf,.jpeg,.jpg,.png}

% Definiciones de colores (para hidelinks)
\definecolor{LightCyan}{rgb}{0,0,0}
\definecolor{Cerulean}{rgb}{0,0,0}
\definecolor{Fuchsia}{rgb}{0,0,0}

% Keywords (español e inglés)
\def\keywordsEn{\vspace{.5em}
{\textbf{\textit{Key words ---}}\,\relax%
}}
\def\endkeywordsEn{\par}

\def\keywordsEs{\vspace{.5em}
{\textbf{\textit{Palabras clave ---}}\,\relax%
}}
\def\endkeywordsEs{\par}


% Abstract (español e inglés)
\def\abstractEs{\vspace{.5em}
{\textbf{\textit{Resumen ---}}\,\relax%
}}
\def\endabstractEs{\par}

\def\abstractEn{\vspace{.5em}
{\textbf{\textit{Abstract ---}}\,\relax%
}}
\def\endabstractEn{\par}

% Estilo páginas de capítulos
\fancypagestyle{plain}{
\fancyhf{}
\fancyfoot[CO]{\footnotesize\emph{\nombretrabajocorto}}
\fancyfoot[RO]{\thepage}
\renewcommand{\footrulewidth}{.6pt}
\renewcommand{\headrulewidth}{0.0pt}
}

% Estilo resto de páginas
\pagestyle{fancy}

% Estilo páginas impares
\fancyfoot[CO]{\footnotesize\emph{\nombretrabajocorto}}
\fancyfoot[RO]{\thepage}
\rhead[]{\leftmark}

% Estilo páginas pares
\fancyfoot[CE]{\emph{\pieparcen}}
\fancyfoot[LE]{\thepage}
%\fancyfoot[RE]{\pieparizq}
\lhead[\leftmark]{}

% Guía del pie de página
\renewcommand{\footrulewidth}{.6pt}

% Nombre de los bloques de código
\renewcommand{\lstlistingname}{Código}

%
% Estilo de los lstlistings
%\lstset{
%    frame=tb,
%    breaklines=true,
%    postbreak=\raisebox{0ex}[0ex][0ex%{\ensuremath{\color{gray}\hookrightarrow\space}}
%}

% Definiciones de funciones para los títulos
\newlength\salto
\setlength{\salto}{3.5ex plus 1ex minus .2ex}
\newlength\resalto
\setlength{\resalto}{2.3ex plus.2ex}

%% Estilo de los acrónimos
\renewcommand{\acronymname}{Glosario}
\renewcommand{\glossaryname}{Glosario}
\pretolerance=2000
\tolerance=3000

% Texto índice de tablas
\addto\captionsspanish{
\def\tablename{Tabla}
\def\listtablename{\'Indice de tablas}
}

% Traducir appendix/appendices
\renewcommand\appendixtocname{Apéndices}
\renewcommand\appendixpagename{Apéndices}

% Comando code (lstlisting sin cambio de página)
\lstnewenvironment{code}[1][]%
  { \noindent\minipage{0.935\linewidth}\medskip
    \vspace{5mm}
    \lstset{basicstyle=\ttfamily\footnotesize,#1}}
  {\endminipage}

% Comando para poder poner imágenes con texto alrededor
\usepackage{wrapfig}


%New colors defined below
\definecolor{codegreen}{rgb}{0,0.6,0}
\definecolor{codegray}{rgb}{0.5,0.5,0.5}
\definecolor{codepurple}{rgb}{0.58,0,0.82}
\definecolor{backcolour}{rgb}{0.95,0.95,0.92}

%Code listing style named "mystyle"
\lstdefinestyle{STM-code}{
  backgroundcolor=\color{backcolour},   
  commentstyle=\color{codegreen},
  keywordstyle=\color{magenta},
  numberstyle=\tiny\color{codegray},
  stringstyle=\color{codepurple},
  basicstyle=\footnotesize,
  breakatwhitespace=false,         
  breaklines=true,                 
  captionpos=b,                    
  keepspaces=true,                 
  numbers=left,                    
  numbersep=5pt,                  
  showspaces=false,                
  showstringspaces=false,
  showtabs=false,                  
  tabsize=2,
  morekeywords = {hint8_t},
}

%"mystyle" code listing set
\lstset{style=STM-code}

\lstset{
     literate=
         {á}{{\'a}}1
         {é}{{\'e}}1
         {í}{{\'i}}1
         {ó}{{\'o}}1
         {ú}{{\'u}}1
         {Á}{{\'A}}1
         {É}{{\'E}}1
         {Í}{{\'I}}1
         {Ó}{{\'O}}1
         {Ú}{{\'U}}1
         {ñ}{{\~n}}1
         {Ñ}{{\~N}}1
}

\usepackage{caption}
\usepackage{subcaption}
\usepackage{multirow}

\usepackage{float} %Para obligar a las figuras a quedarse en un sitio

% Definiciones de comandos
\newcommand{\nombreautor}{Javier Benavides Caro}
\newcommand{\nombredirector}{Juan Manuel López Navarro}
\newcommand{\nombretrabajo}{Desarrollo y puesta en funcionamiento de un sistema de adquisición EEG con capacidades de procesado}
\newcommand{\nombretrabajocorto}{Prototipado de sistema de adquisición EEG}
\newcommand{\fecha}{\today}
\newcommand{\lugar}{Madrid}
\newcommand{\master}{Máster Universitario en Ingeniería de Telecomunicación}
% Descomentar si tu trabajo tiene un co-director
%\newcommand{\nombrecodirector}{TODO: Nombre del co-director}
% Descomentar si tu trabajo está asociado a un grupo de investigación
% \newcommand{\grupoInvestigacion}{TODO: Grupo de investigación}
\newcommand{\departamento}{Grupo de investigación en Intrumentación y Acústica Aplicada}
\newcommand{\facultad}{Escuela Técnica Superior de Ingenieros de Telecomunición}
\newcommand{\universidad}{Universidad Politécnica de Madrid}
\newcommand{\pieparizq}{\nombretrabajocorto}
\newcommand{\pieparcen}{\nombretrabajocorto}
\newcommand{\logoizq}{logo_politecnica}
\newcommand{\logoder}{escudo_teleco}
\newcommand{\correo}{j.bcaro@alumnos.upm.es; javierbenacaro@gmail.com}


% Glosario y acrónimos
%\makeindex
%\makeglossaries
\input{src/glosario}


% Inicio del documento
\begin{document}

% Elección del idioma (español)
\selectlanguage{spanish}

%
% Portada
%
\pagenumbering{gobble}
\include{portada}
\hypersetup{pageanchor=true}

% Estilo de párrafo de los capítulos
\setlength{\parskip}{0.75em}
\renewcommand{\baselinestretch}{1.25}
% Interlineado simple
\spacing{1}

%
% Agradecimientos
%

\include{src/Dedicatoria}  
\chapter*{Agradecimientos}

La realización de este proyecto no habría sido posible sin la ayuda de ciertas personas que me han apoyado durante toda su realización.

En primer lugar al tutor, que ha proporcionado su ayuda en los momentos más complicados y siempre ha mantenido un grado de paciencia y exigencia alto, dando lugar a una calidad en el trabajo sin duda superior a la que habría conseguido alcanzar por mis propios medios.

A Jesús Alonso por su ayuda durante el comienzo del proyecto, asentando unas bases y explicando todos aquellos conceptos desconocidos hasta ese momento.

Por supuesto a mis padres, pues su apoyo y esfuerzo me ha permitido estudiar aquello que más me gusta desde el principio hasta este momento y estoy seguro de que este camino no ha estado exento de sacrificios por su parte.

Merece especial mención Nerea, que ha estado presente en todo momento ayudando y compartiendo sus opiniones siempre que ha sido necesario, sacrificando tiempo libre y horas de sueño sin dudarlo.

Por último a todos mis amigos, tanto de Granada como de Alcobendas.  

%
% Resumen
%
\pagenumbering{roman}
\setcounter{page}{0}
% Resumen en español
\chapter*{Resumen}

\begin{abstractEs}

	No debe superar las 500 palabras.
	
\end{abstractEs}

% Palabras clave en español
\begin{keywordsEs}
	TODO: Palabras clave en español, separadas por coma.
\end{keywordsEs}


% Resumen en inglés
\chapter*{Abstract}

\begin{abstractEn}

	No debe superar las 500 palabras.

\end{abstractEn}

% Palabras clave en inglés
\begin{keywordsEn}
TODO: Palabras clave en inglés, separadas por coma.
\end{keywordsEn}




%
% Glosario
%
%Para que se incluyan debes ejecutar en el terminal el siguiente comando:
%makeindex -s main.ist -t main.glg -o main.gls main.glo

% Para que compile hace falta tener instalado Perl y ejecutar
% makeglosaries.exe main desde powershell/cmd

% Acrónimos

%TODO: Añadir aquí los acrónimos
% Ejemplo de acrónimo
%\newacronym{FPGA}{FPGA}{Field-Programmable Gate Array}
\newacronym{PCB}{PCB}{Print Board Circuit}
\newacronym{EEG}{EEG}{Electroencefalograma}
\newacronym{ECG}{ECG}{Electrocardiograma}
\newacronym{ADC}{ADC}{Convertidores Analógico-Digital}
\newacronym{SPI}{SPI}{Serial Peripheral Interface}
\newacronym{GPIO}{GPIO}{General Purpose Input/Output}
\newacronym{USB}{USB}{Universal Serial Bus}
\newacronym{CPU}{CPU}{central processing unit}
\newacronym{RAM}{RAM}{Random Access Memory}
\newacronym{SoC}{SoC}{System on Chip}
\newacronym{UART}{UART}{Universal Asynchronous Receiver-Transmitter}
\newacronym{IDE}{IDE}{Integrated Development Environment}
\newacronym{SPS}{SPS}{Samples Per Second}
\newacronym{MCU}{MCU}{Microcontroller Unit}
\newacronym{DSP}{DSP}{Digital Signal Processing}
\newacronym{FCPU}{FCPU}{Frecuencia de la CPU}
\newacronym{FPU}{FPU}{Floating Point Unit}
\newacronym{LQFP64}{LQFP64}{ 64 pin Low-profile Quad Flat Package}
\newacronym{OTG}{OTG}{On The Go}
\newacronym{LED}{LED}{Light-Emitting Diode}
\newacronym{PLL}{PLL}{Phase-Locked Loop}
\newacronym{SMT}{SMT}{Surface-Mount Technology}
\newacronym{THT}{THT}{Through-Hole Technology}
\newacronym{HSE}{HSE}{High-Speed External clock}
\newacronym{LSI}{LSI}{Low-Speed Internal clock}
\newacronym{CSS}{CSS}{Clock Security System}
\newacronym{HAL}{HAL}{Hardware Abstraction Layer}
\newacronym{FIR}{FIR}{Finite Impulse Response}

% Glosario

%TODO: Añadir aquí las definiciones del glosario
% Ejemplo de glosario
\newglossaryentry{bitstream}{
	name={bitstream},
	description={En este contexto se refiere al binario que configura el Hardware de la FPGA.}}

\newglossaryentry{Full Duplex}{
	name={Full Duplex},
	description={Cualidad de los elementos que permiten la entrada y salida de datos de forma simultánea.}}

\newglossaryentry{Tensión de Dropout}{
	name={Tensión de Dropout},
	description={Mínima diferencia de tensión entre la entrada y la salida dentro de la cual el circuito es todavía capaz de regular la salida dentro de las especificaciones.}}

\printglossary[title=Glosario,toctitle=Glosario]
\printglossary[title=Acrónimos,toctitle=Acrónimos,type=\acronymtype]


% Estilo de párrafo de los índices
\setlength{\parskip}{1pt}
\renewcommand{\baselinestretch}{1}

%
% Tabla de contenidos
%
\tableofcontents
\listoffigures
\listoftables
\cleardoublepage

% Estilo de párrafo de los capítulos
\setlength{\parskip}{0.75em}
\renewcommand{\baselinestretch}{1.25}
% Interlineado simple
\spacing{1}
% Numeración contenido
\pagenumbering{arabic}
\setcounter{page}{1}

%
% Introducción
%
\chapter{Introducción\label{sec:introduccion}}

TODO: Introducción

Anteproyecto más bonito
%
% Estado del arte
%
\chapter{Estado del Arte\label{sec:EstadoDelArte}}

Con el paso del tiempo se ha demostrado que una de nuestras mayores virtudes como seres humanos es la habilidad de aprovechar el saber cultivado por otras personas para realizar nuevos descubrimientos con mayor facilidad. En la actualidad con la ayuda de Internet esta ventaja se ha visto potenciada hasta límites insospechados.

Como se ha mencionado con anterioridad en el capítulo \ref{sec:introduccion}, a lo largo de los años se han desarrollado numerosas alternativas a los dispositivos presentes en los hospitales y laboratorios utilizados normalmente para el estudio del cerebro. 
\\Aunque se han invertido muchos recursos en estos dispositivos, el objetivo es permitir ampliar el número de personas capaces de estudiar el cerebro humano, consiguiendo  así aumentar las posibilidades de mejorar nuestro conocimiento sobre el mismo.

De esta forma debería ser más fácil realizar nuevos descubrimientos como, por ejemplo, encontrar nuevas formas de diagnosticar enfermedades o de realizar una comunicación hombre-máquina para aquellas personas que por un motivo u otro no pueden utilizar los medios convencionales.

A lo largo de este capítulo se presentarán algunos de los dispositivos capaces de capturar un EEG haciendo uso de electrodos, algunos a nivel personal, otros enfocados a la docencia y, por supuesto, algunos diseñados por empresas con el fin de realizar un producto final que vender a terceros.
\\Aunque éstos dispositivos pueden presentar características muy diversas en función de la persona que los crea y el objetivo del mismo, normalmente se pueden dividir en dos grandes grupos. Dependiendo del tipo de electrodo que se utilice para captar las señales se hablará de dispositivos basados en electrodos húmedos o en electrodos secos.

\section{Tipos de electrodos\label{sec:Tipos_Electrodos}}

Los electrodos hacen la función de interfaz adaptadora entre los distintos medios por los que se transmiten las señales.

Rellenar hablando de como son los electrodos, cómo se suelen usar, alguna imagen donde se vea un electrodo y algún esquema eléctrico.

Comparar los electrodos con la función de los huesos del oído al adaptar la señal acústica para que llegue mejor al oído.

\subsection{Electrodos húmedos\label{sec:Elec_humedos}}

Los electrodos húmedos, 

IMAGEN de ejemplo de electrodos húmedos.

Ventajas e inconvenientes.

\subsection{Electrodos secos\label{sec:Elec_secos}}

IMAGEN de ejemplo de electrodos húmedos.

Ventajas e inconvenientes.

\section{Dispositivos similares\label{sec:Disp_similares}}



\subsection{Proyectos personales\label{sec:Pro_personales}}

En internet se pueden encontrar algunos ejemplos de personas que han dedicado su tiempo a crear dispositivos capaces de captar un EEG...

Hablar de los códigos accesibles en distintas páginas web, github y otros sitios.

Explicar que me he basado en ellos para agilizar el diseño ya que las definiciones de los distintos registros son similares.

\subsection{Proyectos docentes\label{sec:Pro_docentes}}

Hablar del de Nerea, como se ha trabajado con el y en que se basa.

\subsection{Comerciales\label{sec:Pro_empresa}}

Ejemplos comerciales como zanna u otras empresas.

Sistemas BCI comerciales.

Claramente todos los ejemplos anteriores presentan un coste bastante dispar entre ellos costando desde los cientos de euros de los proyectos personales y docentes hasta algunos miles de euros en el caso de productos comerciales.

Historia procesadores (Gráfica de procesadores)
Antecedentes de la placa (Open BCI)

%
% Diseño
%
\chapter{Diseño\label{sec:Diseno}}

Este proyecto consiste en una \textbf{ampliación y cambio de enfoque} de otro proyecto llevado a cabo de manera simultánea por una alumna de la Universidad Politécnica de Madrid llamada Nerea Urrestarazu que, a su vez, se basa en el Kit de demostración de rendimiento del ADS1299 proporcionado por Texas Instrument.

\begin{figure} [h]
    \centering
    \includegraphics[width=7cm]{Placa_Nerea}
    \caption{Placa final del proyecto base}
    \label{fig:Placa_base}
\end{figure}

\section{Diseño base\label{sec:Diseno_base_N}}

El \textbf{proyecto original} consiste en el \textbf{diseño y desarrollo} de una \textbf{placa de adquisición de \acrshort{EEG}} haciendo uso de los integrados ADS1299 junto con un sistema de transmisión hacia el ordenador tanto inalámbricamente como a través de USB. La figura \ref{fig:Diseno_base} muestra las partes que componen dicho diseño.

\begin{figure} [h]
    \centering
    \includegraphics[width=12cm]{Esquema_diseno_base}
    \caption{Esquema del proyecto base}
    \label{fig:Diseno_base}
\end{figure}

\subsection{Adquisición de datos\label{sec:Adquisicion_N}}

La parte encargada de la \textbf{adquisición} está compuesta por un par de \textbf{bancos de electrodos} dispuestos en los laterales de la placa seguidos por un filtro paso-bajo con frecuencia de corte de 6.79kHz, encargado de eliminar las componentes de frecuencias muy altas, no deseadas en el estudio de un \acrshort{EEG}. A continuación se encuentran conectados a sus respectivos bancos los \textbf{\acrshort{ADC} ADS1299}. 
\\Estos convertidores son capaces de adquirir información de forma independiente o en modo ``\textit{Daisy Chain}'' y \textbf{transmitirla a través de \acrshort{SPI}} hacia otros dispositivos cuya misión será gestionarla.

\begin{figure} [h]
    \centering
    \includegraphics[width=6cm]{ADS1299}
    \caption{Convertidor Analógico-Digital ADS1299}
    \label{fig:ADS1299}
\end{figure}

El \acrshort{SPI} presente en el convertidor permite leer todos los registros del ADS y escribir la gran mayoría. Aunque normalmente se leen los relacionados con los datos convertidos, también es posible saber el estado de los \acrshort{GPIO} o el identificador único del dispositivo leyendo su registro asociado.

La \textbf{configuración} del ADS se realiza mediante la \textbf{escritura de ciertos registros}, cada uno asociado a un parámetro específico. La tabla \ref{tab:Conf_Reg_ADS} muestra los registros disponibles, tanto de lectura como de configuración, una descripción básica y la dirección de memoria asociada a los mismos.

\begin{table} [h]
    \centering
    \includegraphics[width=\textwidth]{Tabla_registros_ADS}
    \caption{Tabla de registros de la familia ADS \cite{Datasheet_ADS}}
    \label{tab:Conf_Reg_ADS}
\end{table}

Una descripción más detallada de cada uno de los bits de cada registro se puede encontrar en el \textit{datasheet} del componente \cite{Datasheet_ADS}.

Como se puede ver en la tabla \ref{tab:Conf_Reg_ADS}, los convertidores cuentan con una gran cantidad de opciones de configuración. Para el desarrollo de este proyecto \textbf{se ha implementado un sistema de configuración que permite la lectura y escritura de todos los registros}.

\subsection{Transmisión de datos\label{sec:Transmisión_N}}

Tras completar el proceso de adquisición de datos resulta necesario transmitir dicha información hacia un dispositivo capaz de procesarla. Para ello la placa original contaba con dos alternativas. La primera consiste en, mediante \textbf{\acrshort{USB}} y acopladores aislantes, transmitir la información a un ordenador. 
\\La segunda hace uso de dos tecnologías inalámbricas distintas que funcionan de forma excluyente y son seleccionables con un \textit{jumper}: \textbf{WiFi} o \textbf{Bluetooth}.

\subsubsection{WiFi\label{sec:WiFi_N}}

Para la transmisión de datos a través de \textbf{WiFi} se seleccionó el \textbf{módulo ESP12-E}, basado en el \acrshort{SoC} ESP2866, también conocido nodemcu.
\\Este cuenta con un microcontrolador (\acrshort{MCU}) embebido de 32 bits (Tensilica L106) con una memoria \acrshort{RAM} de 36kB y una velocidad de reloj de la \acrshort{CPU} de hasta 80MHz, proporcionando suficiente potencia para las tareas básicas.

Así mismo se incluye montado en el mismo paquete una memoria flash de 4MB en la que almacenar el código de los programas que se ejecutarán y una antena embebida, dotando al módulo de conectividad en la banda de 2.4GHz.

\begin{figure} [h]
    \centering
    \includegraphics[width=6cm]{ESP8266}
    \caption{ESP8266}
    \label{fig:ESP8266}
\end{figure}

A efectos de diseño es muy importante saber cuales serán las entradas/salidas del dispositivo así como los \textbf{pines dedicados para su programación}. La figura \ref{fig:ESP8266_pinout} muestra un resumen de todas las funciones de cada uno de los pines. Como se puede observar, el \acrshort{SPI} hace uso de los pines 5, 6, 7 y 16. Por otro lado el \acrshort{UART}, necesario para la programación del micro hace uso de los pines 16 y 17. Dichos pines deberán \textbf{reservarse} posteriormente en la fase de diseño de la \acrshort{PCB}.

El dispositivo tiene tres modos de arranque dependiendo del sitio desde el que cargue el código y la selección de uno u otro modo viene determinada por los pines MTDO, GPIO0 y GPIO2. \\La tabla \ref{tab:ESP_Boot_Modes} los distintos modos de arranque y el estado en el que deben estar de los pines para iniciar en dicho modo.
\begin{table} [h]
 	\centering
	\begin{tabular}{|c|c|c|c|c|}
		\hline 
		MTDO & GPIO0 & GPIO2 & Modo & Descripción \\ 
		\hline 
		L & L & H & UART & Descarga el código desde UART \\ 
		\hline 
		L & H & H & Flash & Carga desde memoria Flash a través de SPI \\ 
		\hline 
		H & x & x & SDIO & Carga desde una tarjeta SD \\ 
		\hline 
	\end{tabular} 
	\caption{Modos de arranque del ESP12-E \cite{ESP_Boot_mode}}
    \label{tab:ESP_Boot_Modes}
\end{table}	

\clearpage

\begin{figure} [h]
    \centering
    \includegraphics[width=15cm]{esp8266-esp12e-pinout}
    \caption{Resumen de todas las Entradas/Salidas del ESP12-E \cite{ESP_Pinout}}
    \label{fig:ESP8266_pinout}
\end{figure}

\subsubsection{Bluetooth\label{sec:Bluetooth_N}}

Para la transmisión de datos a través de \textbf{Bluetooth} se seleccionó el módulo Simblee™ RFD77101 ya que al igual que el módulo WiFi, cuenta con interfaces de \textbf{comunicación a través de \acrshort{SPI}} para conectarse con el \acrshort{ADC} (pines 21, 22, 31 y 32) y de \acrshort{UART} para su programación posterior programación (pines 23 y 24).

\begin{figure} [H]
    \centering
    \includegraphics[width=5cm]{BT}
    \caption{Simblee™ RFD77101 \cite{Datasheet_BT}}
    \label{fig:BT}
\end{figure}

El módulo presenta un ARM Cortex-M0 como \acrshort{CPU} con 128KB de memoria Flash, 24KB de \acrshort{RAM} y una frecuencia de reloj de 16MHz. 

Todos los dispositivos de transmisión inalámbrica anteriormente mencionados permiten su \textbf{programación utilizando el \acrshort{IDE} de Arduino} lo cual facilita sensiblemente el proceso de desarrollo y prototipado teniendo la ventaja adicional de que es Software Libre.


\subsubsection{USB\label{sec:USB_N}}


Como este proyecto tiene como objetivo independizar el sistema lo máximo posible del ordenador se ha optado por \textbf{desestimar} el sistema de transmisión por \acrshort{USB} conservando solamente la interfaz inalámbrica. 

\section{Diseño final\label{sec:Diseño_final}}
          
Llegados a este punto se han analizado las características más importantes de cada uno de los elementos presentes en el sistema original, siendo los más importantes las distintas interfaces de comunicación y los pines de programación.
\\Con esta información ya es posible independizar cada uno de esos elementos y crear un \textbf{nuevo diseño que cumpla con las especificaciones de este proyecto}.

\begin{figure} [h]
    \centering
    \includegraphics[width=13cm]{Esquema_proyecto_Javier}
    \caption{Esquema del proyecto final}
    \label{fig:Esquema_proyecto_Javier}
\end{figure}

\clearpage

El nuevo sistema estará compuesto por \textbf{tres partes}:
\begin{itemize}
\item En primer lugar etapa de \textbf{adquisición} compuesta de un \textbf{banco de electrodos} con sus correspondientes filtros analógicos y \textbf{dos \acrshort{ADC}}. 
\item Posteriormente un \textbf{microcontrolador} se encargará de realizar el \textbf{procesado} de la señal y la \textbf{gestión} de los distintos elementos.
\item Por último la transmisión de datos se realizará de forma \textbf{inalámbrica} a un ordenador u otro dispositivo mediante \textbf{Bluetooth o WiFi}.
\end{itemize}

Como se puede observar en la figura \ref{fig:Esquema_proyecto_Javier}, el ordenador sigue estando presente en el sistema pero en esta ocasión su función se limita a mostrar la información siendo fácilmente sustituible en un futuro por un dispositivo menos potente y barato. Al utilizar un microcontrolador en la propia placa de adquisición se consigue aumentar la independencia del sistema y se dota de unas características muy interesantes, tanto de procesado de señal como de almacenaje de la misma o gestión del consumo.

\subsection{Selección del microcontrolador\label{sec:Sel_micro}}

\textbf{El microcontrolador es el núcleo del sistema}. Para poder interactuar con todos los elementos anteriormente descritos deberá contar con las siguientes características:
\begin{itemize}
\item Bajo precio.
\item Bajo consumo.
\item Capacidad procesado de señal.
\item Para poder utilizar el máximo de la velocidad de adquisición de los ADS (16k\acrshort{SPS}) y evitar cuellos de botella, deberá contar con un Bus SPI de al menos 5.85Mb/s dedicados para la transmisión y la misma cantidad para recepción o dos buses.
\[Tasa\ de\ transferencia = 16k\ Muestras/s * 8\ canales * 24bits * 2\ ADS = 5.85Mb/s\]

\end{itemize}

Hay una gran cantidad de microcontroladores en el mercado que podrían utilizarse para este proyecto pero, de entre todos los disponibles, aquellos con arquitectura \textbf{ARM} son los que mejor se adaptan a las especificaciones. Concretamente los más adecuados son aquellos englobados en la familia \textbf{Cortex M4} ya que están especialmente diseñados con \acrshort{DSP} integrado para conseguir un \textbf{rendimiento máximo minimizando el consumo}.

\begin{figure}[H]
  \centering
  \begin{subfigure}[b]{0.49\textwidth}
   	\centering
    \includegraphics [height=1.5cm]{Texas_Instument}
    \caption{Logo de Texas Instument \cite{Texas_Instrument}}
    \label{fig:Logo_TI}
  \end{subfigure}
  \hfill
  \begin{subfigure}[b]{0.49\textwidth}
  	\centering
    \includegraphics[height=1.5cm]{STMicroelectronics}
    \caption{Logo de STMicroelectronics\cite{STMicroelectronic}}
    \label{fig:Logo_STM}
  \end{subfigure}
  \caption{Principales fabricantes contemplados}
\end{figure}

La tabla \ref{tab:Comparativa_MCU} muestra varios dispositivos de esa familia vendidos por Texas Instrument o STMicroelectronics y un resumen de sus características principales.
\begin{table} [h]
	\centering
	\begin{tabular}{|c|c|c|c|c|}
\hline 
 &\multicolumn{2}{c|}{\textbf{Texas Instruments}}& \multicolumn{2}{c|}{\textbf{STMicroelectronics}} \\ 
\hline 
 & TM4C123GH6PM & TM4C1294NCPDT & STM32F405& STM32F469  \\ 
\hline 
\acrshort{FCPU} [MHz] & 80 & 120 & 168 & 180 \\ 
\hline 
Flash [kB] & 256 & 1024 & 1024 & 2048 \\ 
\hline 
\acrshort{RAM} [kB] & 32 & 256 & 192 & 384 \\ 
\hline 
\acrshort{SPI} & x4 & x4 & x2 + 1 & x6 \\ 
\hline 
Precio [€] & 8,72 & 12,74 & 9,24 & 13,48 \\ 
\hline 
	\end{tabular} 
	\caption{Comparativa entre distintos MCUs}
	\label{tab:Comparativa_MCU}
\end{table}

Todos los dispositivos anteriormente contemplados tienen \acrshort{DSP} integrados así como una Unidad de Punto Flotante (\acrshort{FPU}) que permite realizar \textbf{operaciones matemáticas avanzadas} de forma óptima.

Entre los dispositivos anteriores, los pertenecientes a la familia \textbf{STM} presentan mejor relación coste/prestaciones, pero el verdadero factor diferenciador son las herramientas dispuestas para la comunidad por parte del fabricante.
\\En la página web se pueden encontrar distintas \textbf{aplicaciones} y \textbf{documentación} que facilitan sensiblemente el proceso de desarrollo para esta plataforma. 

Finalmente se seleccionó el \textbf{\acrshort{MCU} STM32F405} en el formato \textbf{\acrshort{LQFP64}} ya que sus características se ajustan perfectamente a las especificaciones, manteniendo unas muy buenas prestaciones y un precio bastante bajo. Todo esto con el valor añadido de que ya se contaba con la placa de desarrollo \textbf{STM32F4 Discovery} lo cual permitió comenzar con el aprendizaje y estudio del entorno sin la necesidad de esperar al diseño, impresión y soldado de la placa final.

\begin{figure} [h]
    \centering
    \includegraphics[width=5cm]{STM32F4_Discovery}
    \caption{Placa de desarrollo STM32F4 Discovery}
    \label{fig:STM32F4_Discovery}
\end{figure}

El \acrshort{MCU} cuenta con 3 buses SPI independientes que pueden funcionar en modo \textit{\gls{Full Duplex}}. El SPI$_1$  es capaz de funcionar hasta 42Mb/s mientras que los SPI$_2$ y SPI$_3$ pueden comunicar información hasta 21Mb/s. Los pines dedicados a dichos buses se pueden consultar en el \textit{Datasheet} del componente \cite{Datasheet_STM}.
\\Otra característica interesante de este \acrshort{MCU} es la presencia de forma nativa de un gestor de \acrshort{USB} \acrlong{OTG} (\acrshort{OTG}) permitiendo así conectar un dispositivo \acrshort{USB} para almacenar información a largo plazo.

\clearpage

\subsection{Alimentación\label{sec:Alimentación}}

Tras seleccionar todos los elementos se va a proceder a escoger una alimentación que permita a todos los dispositivos funcionar en condiciones óptimas.
\\La tabla \ref{tab:Alimentacion} muestra un resumen de todos los elementos presentes en el sistema junto con los rangos de voltaje recomendados por los fabricantes para su alimentación.

\begin{table} [h]
	\centering
	\begin{tabular}{|c|c|c|}
	\hline 
	Dispositivo & V$_{\text{min}}$ [V] & V$_{\text{max}}$ [V] \\ 
	\hline 
	Simblee BT & 1.8 & 3.6 \\ 
	\hline 
	ESP12-E & 3.0 & 3.6 \\ 
	\hline 
	STM & 1.8 & 3.6 \\ 
	\hline 
	ADS (Digital) & 1.8 & 3.6 \\ 
	\hline 
	ADS (Analógica) & 4.75 & 5.25 \\ 
	\hline 
	USB & 5 & 5 \\ 
	\hline 
	\end{tabular} 
	\caption{Rangos de alimentación de todos los elementos}
	\label{tab:Alimentacion}
\end{table}

De la tabla \ref{tab:Alimentacion} se deduce que para que todos los dispositivos funcionen correctamente \textbf{será necesario dotar a la placa de 5 voltios} con los que se podrá alimentar la parte analógica del ADS así como el USB mientras que la parte digital se puede alimentar en el rango de 3V a 3.6V. 

Por motivos de compatibilidad con el diseño anterior y tras comprobar que se cumple con los requisitos impuestos por los nuevos elementos del sistema se ha optado por mantener el mismo esquema de alimentación que en el proyecto base.

El sistema de alimentación se compone de dos partes principales, cada una encargada de proporcionar el voltaje deseado manteniendo el ruido generado por el mismo al mínimo.

\subsubsection{Alimentación 3.3V\label{sec:Alimentacion_3.3V}}
Para conseguir un voltaje de \textbf{3.3V} estable se ha utilizado el regulador \textbf{AZ1117C-3.3} ya que es capaz de conseguir una precisión para el voltaje de salida del $\pm$1\% así como un ruido de salida de 0.003\% V$_{\text{out}}$ entre 10Hz y 10kHz.
En la figura \ref{fig:Alim_3.3} se muestra el esquema eléctrico utilizado, que es una variación del circuito recomendado por el fabricante en su \textit{datasheet} \cite{Datasheet_3.3}.

\begin{figure} [h]
    \centering
    \includegraphics[width=10cm]{Alim_3_3}
    \caption{Esquema de alimentación a 3.3V}
    \label{fig:Alim_3.3}
\end{figure}

\subsubsection{Alimentación 5V\label{sec:Alimentacion_5V}}

Los 5V serán utilizados por el dispositivo de almacenamiento USB para su alimentación y por el ADS para generar los distintos voltajes de referencia que necesita para operar correctamente. Al afectar de forma directa a las mediciones realizadas por el ADS la eliminación de variaciones en este voltaje es crucial, pues estas supondrán un ruido añadido a la señal final.

El regulador encargado de proporcionar \textbf{5V} es el \textbf{MCP1711}. Este integrado está caracterizado por tener un rizado de salida menor al 1\% del voltaje de salida así como de poseer una corriente en reposo muy baja. Este último parámetro facilitará el diseño de un sistema portátil alimentado por baterías aumentando la duración de la misma. La figura \ref{fig:Alim_5} muestra el esquema eléctrico utilizado.

\begin{figure} [h]
    \centering
    \includegraphics[width=10cm]{Alim_5}
    \caption{Esquema de alimentación a 5V}
    \label{fig:Alim_5}
\end{figure}

Para funcionar correctamente ambos integrados necesitan que el voltaje de entrada sea superior al voltaje de salida en un factor que en la documentación técnica recibe el nombre de ``\gls{Tension de Dropout}''(V$_{\text{DROP}}$).
\\De acuerdo al \textit{datasheet} de ambos reguladores, V$_{\text{DROP}}$ es para 3.3V y 5V, 1.3V y 0.43V respectivamente. Con esa información se deduce que el integrado que limita el diseño es el regulador de 5V de modo que con una entrada de al menos 5.43V ambos reguladores deberían funcionar correctamente.

Finalmente se ha optado por una \textbf{fuente de alimentación de 6V} ya que esto permitirá hacer uso de pilas o baterías para alimentar el sistema, dotándolo de independencia de la red eléctrica y eliminando el riesgo de electrocución.

\clearpage

\section{Circuito electrónico y esquemáticos\label{sec:Esquemáticos}}

El último paso de la fase de desarrollo será generar un \textbf{esquemático} capaz de englobar todos los elementos anteriormente presentados, interconectarlos y dar como resultado un sistema funcional.

En este punto es importante valorar dos alternativas de diseño, cada una con sus ventajas e inconvenientes: implementación de todo el circuito de cero o crear una placa que se conecte a la ya existente.

Si bien es cierto que para un diseño final crear una placa que englobe todos los componentes sería lo ideal, pues presentaría un formato más compacto y mejor presentación, hacerlo también supone crear una placa más grande y desaprovechar aquellas ya construidas en proyectos anteriores.

Teniendo en cuenta que se cuenta con varias tarjetas ya montadas y que el sistema está en fase de prototipo, se va a optar por la segunda opción, creando una \textbf{segunda tarjeta independiente} en la que se incluirán todos los elementos correspondientes a la gestión de las señales digitales del sistema, dejando las señales analógicas en la otra tarjeta. De esta forma la fase de diseño de la \acrshort{PCB} y de montaje se simplifica considerablemente y se \textbf{abaratan costes al reutilizarse componentes}.

Para la conexión con la otra tarjeta se aprovechará el espacio dejado por el módulo ESP12-E, pues para comunicarse con el ADS sólo es necesario el bus \acrshort{SPI} y todas las señales necesarias se encuentran accesibles desde los conectores de dicho módulo.

El sistema embebido en esta placa incluirá los elementos que se pueden ver en la figura \ref{fig:Esquematico_global}

\begin{figure} [h]
    \centering
    \includegraphics[width=14cm]{Esquematico_global}
    \caption{Esquema general del sistema}
    \label{fig:Esquematico_global}
\end{figure}

Se ha incluido en la placa un \textbf{botón de reinicio} con su correspondiente circuito electrónico así como una realimentación desde la alimentación hasta uno de los pines \acrshort{ADC} el microcontrolador para poder \textbf{monitorizar el estado de la batería}.

\subsection{Circuito de alimentación\label{sec:Esquemáticos_alim}}

Aunando los dos circuitos presentados en la sección \ref{sec:Alimentación} el resultado final es el obtenido en la figura \ref{fig:Alim_final}.

\begin{figure} [h]
    \centering
    \includegraphics[width=14cm]{Alim_final}
    \caption{Esquemático final del circuito de alimentación}
    \label{fig:Alim_final}
\end{figure}

El \textbf{fusible} (F1) garantiza que la máxima corriente que consumirá el dispositivo es de 500mA, \textbf{evitando} así que el sistema o el paciente sufran \textbf{daños} en caso de un cortocircuito. Adicionalmente se ha incluido un diodo \acrshort{LED} cuya función es indicar el estado de la placa. Si la placa ha encendido correctamente o si se encuentra en funcionamiento el \acrshort{LED} se encenderá.

El diodo (D2) junto con el condensador (C3) evitarán que los transitorios afecten al voltaje de entrada asegurando así que la alimentación que recibirán ambos integrados será lo más estable posible.

Con el objetivo de \textbf{monitorizar el estado de la batería} se ha utilizado el \textbf{\acrshort{ADC} incorporado en el propio microcontrolador}. Como la señal de entrada se encuentra fuera del rango soportado en las especificaciones del STM se ha optado por incorporar las resistencias R2 y R3 formando un divisor de tensión utilizando así la señal resultante para realizar las medidas.

\clearpage

\subsection{Microcontrolador\label{sec:Esquemáticos_micro}}

El \textbf{microcontrolador es el núcleo del sistema}. Este hace de \textbf{centro de control} de todas las señales digitales que se transmiten así como de \textbf{gestor de dispositivos}, decidiendo que dispositivos se encuentran activos en cada momento. 
Para gestionar que dispositivos se encuentran habilitados se ha sustituido el \textit{jumper} de la placa original por \acrshort{GPIO} del microcontrolador. De esta forma se consigue mayor flexibilidad, automatización y se optimiza el consumo. 

Adicionalmente se ha integrado un \acrshort{LED} conectado al pin PA9 que dotará a la placa de indicadores visuales del estado en el que se encuentra.

La figura \ref{fig:Esquematico_micro} muestra una representación del integrado junto con todos los elementos con los que está conectado.

\begin{figure} [h]
    \centering
    \includegraphics[width=\textwidth]{Esquematico_micro}
    \caption{Esquemático del microcontrolador}
    \label{fig:Esquematico_micro}
\end{figure}

El \textbf{modo de arranque} del dispositivo viene determinado por el estado de los pines Boot0 y Boot1. Haciendo uso de las resistencias R13, R14, R15 y R16 se consigue forzar que en condiciones normales de operación el STM \textbf{arranque desde la memoria Flash integrada}. 
\\Con el objetivo de \textbf{reprogramar el microcontrolador} se ha añadido el conector P5. Dicho conector permite alternar entre los distintos modos de arranque del STM e interactuar con el por UART. Esta última característica será la que permitirá subir el código a ejecutar pero también brinda funciones de \textit{debug}.

\clearpage

El sistema aprovecha dos de los tres buses \acrshort{SPI}. El bus SPI1, capaz de transmitir a 41Mb/s, se ha reservado para la comunicación ESP12-E $\Longleftrightarrow$ STM32F4 mientras que el utilizado para comunicarse con los ADS es el SPI2, con una velocidad de hasta 21Mb/s.

Aunque el \acrshort{MCU} puede trabajar con un oscilador interno, se ha optado por la utilización de un \textbf{oscilador externo basado en un cristal de cuarzo}. Esa opción permite una mayor precisión en el reloj y la posibilidad de usar \acrshort{PLL} para aumentar la frecuencia de trabajo del procesador.
\\El circuito asociado al cristal se puede observar en la parte superior izquierda de la figura \ref{fig:Esquematico_micro} pero se incluye a continuación para facilitar la lectura de este documento:

\begin{figure} [h]
    \centering
    \includegraphics[width=10cm]{Detalle_cristal}
    \caption{Detalle del circuito asociado al oscilador externo}
    \label{fig:Detalle_cristal}
\end{figure}


Para realizar un diseño óptimo del circuito del oscilador se ha utilizado como orientación la hoja de características del dispositivo junto con una guía de buenas prácticas \cite{Guia_Oscilador}, ambos proporcionados por el fabricante. La configuración utilizada a nivel de software para la gestión de dicho reloj se explicará en mayor profundidad en el capítulo \ref{sec:Implementacion_soft}.

Se han añadido todos los condensadores recomendados por el fabricante. Algunos deben tener una capacidad determinada en función del modo de funcionamiento del microcontrolador (C25 y C26), otros, denominados \textbf{condensadores de desacoplo}, tienen como objetivo eliminar el ruido de altas frecuencias de la zona de alimentación. Un condensador a resaltar es el C11, denominado \textbf{\textit{Bulk}} y su función es garantizar una alimentación lo más estable posible.

\begin{figure} [h]
    \centering
    \includegraphics[width=10cm]{Condensadores_micro}
    \caption{Condensadores Bulk (C11) y de desacoplo}
    \label{fig:Condensadores_micro}
\end{figure}

 Por simplicidad y legibilidad se han agrupado todos los condensadores en una zona del esquemático, pero a la hora de diseñar la PCB será necesario tener presente que para que el dispositivo funcione correctamente estos últimos \textbf{deben localizarse lo más cerca posible de los pines de alimentación}.

Finalmente, como el \acrshort{MCU} tiene la capacidad de interactuar directamente con dispositivos \acrshort{USB}, con el objetivo de implementar en un futuro características que aprovechen dicha capacidad se ha añadido un conector \acrshort{USB} cuyo bus de alimentación se encuentra controlado por el propio \acrshort{MCU}. De esta forma es posible habilitar y deshabilitar el dispositivo USB y reducir el consumo.

\subsection{Interfaz inalámbrica\label{sec:Esquematico_inalambrica}}

Para finalizar con la parte del diseño, se incluirá una explicación de los esquemáticos necesarios para hacer funcionar los dos microcontroladores encargados de transmitir la información a través de Wifi y Bluetooth.

\subsubsection{ESP12-E\label{sec:Esquematico_ESP}}

El ESP necesita, al igual que el STM, condensadores Bulk y de desacoplo. Ambos se pueden apreciar en la figura \ref{fig:Esquematico_ESP} en la parte inferior izquierda y deberán estar localizados en la \acrshort{PCB} lo más cerca posible de los pines de alimentación.

\begin{figure} [h]
    \centering
    \includegraphics[width=\textwidth]{Esquematico_ESP}
    \caption{Esquemático del ESP-12E}
    \label{fig:Esquematico_ESP}
\end{figure}

La resistencia R10 fuerza al ESP a estar en un estado activo salvo que el STM (Máster del sistema) lo deshabilite haciendo uso del pin ``EN\_WIFI''. 
\\El modo de arranque del ESP viene definido por la resistencia R8 junto con el conector P4. Para poder ponerlo en modo programación sólo será necesario cortocircuitar los pines 6 y 8. El acceso al bus UART se realizará a través de los pines 3 y 5.

\subsubsection{Bluetooth Simblee\label{sec:Esquematico_BT}}

Para el dispositivo Bluetooth se ha incluido otro conector de programación. En esta ocasión sólo serán necesarios los pines 2, 3 y 4 para la programación ya que el \acrshort{IDE} de Arduino se encargará de la gestión de los diferentes modos de arranque el dispositivo.

\begin{figure} [h]
    \centering
    \includegraphics[width=14cm]{Esquematico_BT}
    \caption{Esquemático del dispositivo Bluetooth Simblee}
    \label{fig:Esquematico_BT}
\end{figure}

El bus \acrshort{SPI} así como los de DRDY y START han sido conectados al STM ya que dichas conexiones serán necesarias para la transmisión de datos entre el STM y el módulo Bluetooth.

En esta ocasión se ha dejado sólo un condensador de desacoplo dado que los de \textit{Bulk} dispuestos para los otros dispositivos suplen la necesidad de utilizar otro para este.

Todos los esquemáticos presentados a lo largo de este capítulo han sido generados haciendo uso de la herramienta KiCad. Al ser un esquemático, su lectura e interpretación es independiente de la herramienta siendo el elemento ``PWR\_FLAG'' (presente en algunas figuras) el único exclusivo de dicha herramienta. Este elemento sive para evitar errores, ya que todos los pines categorizados como pines de alimentación que no lleven un ``PWR\_FLAG'' provocarán una alerta al compilar y generar el esquemático. 

%
% Implementación
%
\chapter{Implementación\label{sec:Implementacion}}

TODO: Implementación

Hardware
	PCB (Añadir BOM)
	
	Primeras pruebas y programación
	

Software
	Comunicación ADS - STM
	
	Comunicación STM- ESP
	
	Arduino (Comunicación ESP - PC)
	
	

%
% Resultados
%
\include{src/5_Resultados}

%
% Conclusiones
%
\include{src/6_Conclusiones} 

%
% Bibliografía
%

\include{src/Bibliografia}

%
% Página en blanco
%
\cleardoublepage

%
% Bibliografía
\printbibliography[heading=bibintoc]

% No expandir elementos para llenar toda la página
\raggedbottom

%
% Apéndices
%
\appendix
\cleardoublepage
\addappheadtotoc
\appendixpage

%
% TODO: Apéndices del TFM
%
\chapter{Bill of Materials (BOM)\label{sec:BOM}}

A continuación, en la tabla \ref{tab:BOM} se detalla el desglose de todos los componentes necesarios para el desarrollo de este proyecto. Este documento recibe comunmente el nombre de \textit{Bill Of Materials} y permite realizar una estimación aproximada del coste del producto final.

\begin{table} [H]
\centering
\includegraphics[width = 16cm]{BOM}
\caption{BOM - Bill of Materials}
\label{tab:BOM}
\end{table}

El coste final contando exclusivamente los materiales asciende a 40,49€. 

\clearpage

Es importante destacar que el coste de la tarjeta de procesado no sustituye al de la tarjeta de adquisición sino que lo complementa. Si se desea recrear este proyecto desde cero será imprescindible comprar todos los componentes y montar ambas tarjetas.
Teniendo en cuenta el presupuesto proyectado para la tarjeta de adquisición (97€), el de la tarjeta de procesado (40,49€) y que ciertos componentes como los módulos Bluetooth, WiFi y USB de la tarjeta de adquisición no son necesarios, \textbf{el precio final del sistema completo es inferior a los 113€}.

\chapter{Impacto ético, económico, social y ambiental\label{sec:Impacto}}

El objetivo principal de este proyecto consiste en construir un sistema capaz de realizar la adquisición de un \acrshort{EEG} con mejores prestaciones y un coste inferior al de aquellos disponibles en el mercado.

Tras cumplir este objetivo y liberar los esquemáticos diseñados se sientan unas bases que cualquier persona o institución con menos recursos puede aprovechar para construir y mejorar su propio sistema. De esta forma se amplía sensiblemente el número de personas que puede realizar investigaciones relacionadas con el estudio de un EEG y, con ello, las posibilidades de llegar a conocer en un futuro el funcionamiento de nuestro cerebro.

La investigación en este campo a corto plazo puede ayudar a la detección de patrones que estén asociados a enfermedades o incluso podría permitir realizar interfaces cerebro-ordenador (BCI) que faciliten la vida a aquellas personas que sufren algún tipo de discapacidad motora o mental.

Este proyecto cumple con la directiva RoHS (\textit{Restriction of Hazardous Substances}), es decir, las sustancias peligrosas comúnmente presentes presentes en los distintos componentes electrónicos utilizados durante la realización de este proyecto se encuentran dentro de unos márgenes definidos por la Comunidad Europea.

\begin{figure} [h]
    \centering
    \includegraphics[width=4cm]{rohs}
    \caption{Este proyecto cumple con la directiva RoHS \cite{rohs}}
    \label{fig:rohs}
\end{figure}

\chapter{Esquemático del proyecto\label{sec:Schematic}}
\includepdf[pages={-},angle=-90]{src/bci-base_v4.pdf} % Incluye todas las páginas de un PDF giradas -90º

\chapter{Firmware del ESP12-E\label{sec:Apendice_Code_ESP}}

Debido a la nueva normativa que permite la presentación de la memoria y documentación en formato digital se ha considerado oportuno la inclusión del código desarrollado para este proyecto en este documento, pues puede servir de referencia para futuros trabajos similares. Sin embargo, teniendo en cuenta su gran extensión, si se desea imprimir este documento se recomienda omitir la impresión de aquellos anexos relacionados con el código, especialmente el anexo \ref{sec:Apendice_Code_STM}.

Este apéndice recoge el código desarrollado para el ESP12-E. Incluye todo lo necesario para hacer de interfaz entre el ordenador y el microcontrolador STM32F4.

\begin{lstlisting}[label=algoritmo:ESP:master_fast.ino,style = STM-code,frame=single,caption=ESP:master\_fast.ino]

/*

    GPIO      Name  |   STM
   ==============================
     12       MISO  |   PA7
     13       MOSI  |   PA6
     14       SCK   |   PA5
     15       SS    |   PA4

*/
#include <SPI.h>
#include <ESP8266WiFi.h>
#include <WiFiUdp.h>

// Funciones prototipo
uint8_t read_command(uint8_t command);
void FloatArrayFromSTM(float F_array [], int N_elementos);
float floatFromSTM(void);
uint8_t String2int (String Array);
void Float2TCP (float Float_data, WiFiClient client);

char ssid[] = "SSID";
char pass[] = "PASSWORD";    

IPAddress local_IP(192,168,4,22);
IPAddress gateway(192,168,4,9);
IPAddress subnet(255,255,255,0);
IPAddress remote_IP(192,168,4,121);

// Create an instance of the server
// specify the port to listen on as an argument
WiFiServer server(80);
WiFiUDP Udp;

unsigned int localPort = 2390;      // local port to listen on


// Definición de los pines
//      Función   GPIO#
#define HSPI_CS   15
#define HSPI_CLK  14 
#define HSPI_MISO 12 
#define HSPI_MOSI 13 
#define DRDY_N    4 
#define START     5 
#define LED       2 

#define hotspot false  // hotspot = true
                    // wifi    = false
#define debug true

int i=1;

// la rutina de setup corre una vez o cuando se presiona reset
void setup() {
  Serial.begin(115200);                
  SPI.begin();

  //SPI.beginTransaction(SPISettings(16000000, MSBFIRST, SPI_MODE1));

  SPI.setClockDivider(SPI_CLOCK_DIV2); //Divides 16MHz clock by 2 to set CLK speed to 4MHz
  SPI.setDataMode(SPI_MODE1);  //clock polarity = 0; clock phase = 1 (pg. 8)
  SPI.setBitOrder(MSBFIRST);  //data format is MSB (pg. 25)  
  
  pinMode(HSPI_CS,OUTPUT);
  pinMode(HSPI_CLK,SPECIAL);
  pinMode(HSPI_MISO,SPECIAL);
  pinMode(HSPI_MOSI,SPECIAL);
  pinMode(DRDY_N,INPUT);
  pinMode(START,OUTPUT);
  pinMode(LED,OUTPUT);

  Serial.println();

  #if (hotspot == false)

  // We start by connecting to a WiFi network
  Serial.print("Connecting to ");
  Serial.println(ssid);
  WiFi.begin(ssid, pass);
  
  while (WiFi.status() != WL_CONNECTED) {
    delay(500);
    Serial.print(".");
  }
  Serial.println("");
  
  Serial.println("WiFi connected");
  Serial.println("IP address: ");
  Serial.println(WiFi.localIP());
#else

  Serial.print("Setting soft-AP configuration ... ");
  Serial.println(WiFi.softAPConfig(local_IP, gateway, subnet) ? "Ready" : "Failed!");

  Serial.print("Setting soft-AP ... ");
  Serial.println(WiFi.softAP("ESPBCI_WIFI", "Password_01", false) ? "Ready" : "Failed!");

  Serial.print("Soft-AP IP address = ");
  Serial.println(WiFi.softAPIP());

#endif

  // Start the server
  Serial.println("Starting servers");
  server.begin();
  Udp.begin(localPort);
  Serial.println("Server started");
  Serial.print("Local port: ");
  Serial.println(Udp.localPort());

}

// la rutina loop corre constantemente
void loop() {
    uint8_t command2STM = 0x02;
    uint8_t response = 0x00;
    float F_data [8];
    float F_data_array[250];
    int N_elementos = 250;
    long t1 = 0;
    int offset_configurar = 0x00;
    

  //########### Server handler #################
  // Check if a client has connected
  WiFiClient client = server.available();
  if (!client) {
    //Serial.println("No client available");
    return;
  }
  
  // Wait until the client sends some data
  Serial.println("new client");
  t1 = millis();
  while(!client.available()){
    delay(1);
    Serial.print(".");
    ESP.wdtFeed();    // Evita reinicios por watchdog
    if (millis()-t1 >= 10000){
      Serial.println("Timeout");
      return;
    }
  }
  Serial.println();
  
  // Read the first line of the request
  String req = client.readStringUntil('\r\n');
      
  #if (debug == true)    
    Serial.println(req);
  #endif

  
  // Match the request

  if (req.indexOf("Configurar") != -1)
  {
    command2STM = 0xF0;
    Serial.println("Configurar");
  }
  else if (req.indexOf("Single float") != -1)
   {
    command2STM = 0x01;
    Serial.println("Single float");
  }
  else if (req.indexOf("Float array") != -1)
    {
    command2STM = 0x02;
    Serial.println("Float array");
  }
  else{
    Serial.println("invalid request");
    client.print("invalid request\r\n");
    delay(1);
    client.flush();
    //client.stop();
    return;
  }
  

  // Prepara la respuesta para LabView
  String s = "";
  //client.setNoDelay(true);

  while (response == 0x00) // Espera a que el STM se sincronice con el ESP
  {
    response = read_command(command2STM);
    ESP.wdtFeed();    // Evita reinicios por watchdog
    Serial.println(response,HEX);
  }
   
    #if (debug == true)
      Serial.print("Comando recibido: ");
      Serial.println(response,HEX);
    #endif
    
    switch(command2STM){
      
      case 0xF0:
        #if (debug == true)
          Serial.println("CASO 0: Configurar");
        #endif
        offset_configurar = req.indexOf("Configurar");
        read_command(String2int(req.substring(offset_configurar+11,offset_configurar+19))); //Config 1
        read_command(String2int(req.substring(offset_configurar+20,offset_configurar+28))); //Config 2
        read_command(String2int(req.substring(offset_configurar+29,offset_configurar+37))); //Config 3
        read_command(String2int(req.substring(offset_configurar+38,offset_configurar+46))); //Config 4

        read_command(String2int(req.substring(offset_configurar+47,offset_configurar+55))); //Channel 1
        read_command(String2int(req.substring(offset_configurar+56,offset_configurar+64))); //Channel 2
        read_command(String2int(req.substring(offset_configurar+65,offset_configurar+73))); //Channel 3
        read_command(String2int(req.substring(offset_configurar+74,offset_configurar+82))); //Channel 4
        read_command(String2int(req.substring(offset_configurar+83,offset_configurar+91))); //Channel 5
        read_command(String2int(req.substring(offset_configurar+92,offset_configurar+100))); //Channel 6
        read_command(String2int(req.substring(offset_configurar+101,offset_configurar+109))); //Channel 7
        read_command(String2int(req.substring(offset_configurar+110,offset_configurar+118))); //Channel 8
        
        #if (debug == true)
          Serial.println("CASO 0: END");
        #endif
        
      break;
      case 0x01:    // El STM va a transmitir un dato de tipo float
        #if (debug == true)
          Serial.println("CASO 1: Single Float");
        #endif
        for (int i = 0; i<8; i++)
        {
        F_data[i] = floatFromSTM();
        Serial.println(F_data[i],7);
        }
        for (int i = 0; i<8; i++)
        {
          Float2TCP(F_data[i], client);
        }
        client.write("\r\n");   //cierra la conexión con Labview
        
        #if (debug == true)
          Serial.println("CASO 1: END");
        #endif        
      break;
      case 0x02:
        #if (debug == true)
          Serial.println("CASO 2: ARRAY");
        #endif
        t1 = millis();
        FloatArrayFromSTM(F_data_array, N_elementos);

        for (i=0; i<250; i++)
        {
         Serial.println(F_data_array[i],7);
        }
        
        for (i=0; i<250; i++)
        {
          Float2TCP(F_data_array[i], client);
          
        }
        client.write("\r\n");   //cierra la conexión con Labview

        Serial.println("END - CASO 2: ARRAY");
        
        #if (debug == true)
          Serial.println("CASO 2: END");
        #endif        
        
      break;
      default:
        #if (debug == true)
          Serial.println("Default");
        #endif
        ESP.wdtFeed();    // Evita reinicios por watchdog

        client.write("\r\n");   //cierra la conexión con Labview
        
        #if (debug == true)
          Serial.println("Default: END");
        #endif  
        
      break;
    }

  Serial.println("Client disconnected");
  
  client.flush();
  
}





uint8_t read_command(uint8_t command)
{ 
while (digitalRead(DRDY_N)!=LOW)
{
  ESP.wdtFeed();    // Evita reinicios por watchdog
  //Serial.println("Waiting for DRDY");
}

  digitalWrite(HSPI_CS, LOW);
  command = SPI.transfer(command);
  digitalWrite(HSPI_CS, HIGH);
  while(digitalRead(DRDY_N)==LOW){}  // Bloquea hasta que el ESP considera terminada la transmisión

  return command;
}

void FloatArrayFromSTM(float F_array [], int N_elementos)
{
 union miDato{
  struct
  {
    byte  b[4];     // Array de bytes de tamaño igual al tamaño de la primera variable: int = 2 bytes, float = 4 bytes
    }split;
    float fval;
  } F_recibido; 
  F_recibido.fval = 8765.4321;
  int i = 0;

while (digitalRead(DRDY_N)!=LOW)
{
  ESP.wdtFeed();    // Evita reinicios por watchdog
  //Serial.println("Waiting for DRDY");
}

long t1 = millis();
     digitalWrite(HSPI_CS, LOW);
     for (i = 0; i<N_elementos; i++)
        {
           F_recibido.split.b[0] = SPI.transfer(0b01000000);
           F_recibido.split.b[1] = SPI.transfer(0b01000000);
           F_recibido.split.b[2] = SPI.transfer(0b01000000);
           F_recibido.split.b[3] = SPI.transfer(0b01000000);
           
           F_array[i]=F_recibido.fval;
        }

Serial.println(millis()-t1);
        
     digitalWrite(HSPI_CS, HIGH);
     
while (digitalRead(DRDY_N)==LOW)
{
  ESP.wdtFeed();    // Evita reinicios por watchdog
} 
 // Espera a que DRDY levante, significa que se ha acabado el bucle.
 // Bloquea el micro pero asegura que no entra más de una vez
      
     if (digitalRead(LED)==1)
       digitalWrite(LED,LOW);
     else
       digitalWrite(LED,HIGH);     
}

float floatFromSTM(void)
{
  union miDato{
    struct
    {
      byte  b[4];     // Array de bytes de tamaño igual al tamaño de la primera variable: int = 2 bytes, float = 4 bytes
    }split;
    float fval;
  } F_recibido;
   
  F_recibido.fval = 8765.4321;
  int i = 0;

  digitalWrite(HSPI_CS, LOW);
  while (digitalRead(DRDY_N)!=LOW)
  {
    ESP.wdtFeed();    // Evita reinicios por watchdog
    //Serial.println("Waiting for DRDY");
  }
  F_recibido.split.b[0] = SPI.transfer(0b01000000);
  F_recibido.split.b[1] = SPI.transfer(0b01000000);
  F_recibido.split.b[2] = SPI.transfer(0b01000000);
  F_recibido.split.b[3] = SPI.transfer(0b01000000);
  
  digitalWrite(HSPI_CS, HIGH);
       
  while (digitalRead(DRDY_N)==LOW)
  {
    ESP.wdtFeed();    // Evita reinicios por watchdog
  } 
  // Espera a que DRDY levante, significa que se ha acabado el bucle.
  // Bloquea el micro pero asegura que no entra más de una vez
      
  if (digitalRead(LED)==1)
   digitalWrite(LED,LOW);
  else
   digitalWrite(LED,HIGH);   

  return F_recibido.fval;
}

uint8_t String2int (String Array)
{
  uint8_t output = 0x00;
  for (int i = 0; i < 8; i++)
    bitWrite(output, i, bitRead(Array[7-i], 0));
  
  return output;
}

void Float2TCP (float Float_data, WiFiClient client)
{
  union miDato{
  struct
  {
    byte  b[4];     // Array de bytes de tamaño igual al tamaño de la primera variable: int = 2 bytes, float = 4 bytes
  }split;
  float fval;
} F_recibido;

  F_recibido.fval = Float_data;
 
  client.write(F_recibido.split.b[3]); // Envía la respuesta a LabView
  client.write(F_recibido.split.b[2]); // Envía la respuesta a LabView
  client.write(F_recibido.split.b[1]); // Envía la respuesta a LabView
  client.write(F_recibido.split.b[0]); // Envía la respuesta a LabView*/
}
\end{lstlisting}

\chapter{Firmware del STM32F4\label{sec:Apendice_Code_STM}}

A continuación se incluye el firmware desarrollado para el microcontrolador STM32F4, mostrando los ficheros más importantes y obviando aquellos generados de forma automática o presentes en las librerías aunque estos hayan sido de utilidad.

\begin{lstlisting}[label=algoritmo:STM32F4:main.c,style = STM-code,frame=single,caption=STM32F4:main.c]
/* 
  ************************************************
  * File Name          : main.c
  * Description        : Main program body
  ************************************************
  * This notice applies to any and all portions of this file
  * that are not between comment pairs USER CODE BEGIN and
  * USER CODE END. Other portions of this file, whether 
  * inserted by the user or by software development tools
  * are owned by their respective copyright owners.
  *
  * Copyright (c) 2018 STMicroelectronics International N.V. 
  * All rights reserved.
  *
  * Redistribution and use in source and binary forms, with or without 
  * modification, are permitted, provided that the following conditions are met:
  *
  * 1. Redistribution of source code must retain the above copyright notice, 
  *    this list of conditions and the following disclaimer.
  * 2. Redistributions in binary form must reproduce the above copyright notice,
  *    this list of conditions and the following disclaimer in the documentation
  *    and/or other materials provided with the distribution.
  * 3. Neither the name of STMicroelectronics nor the names of other 
  *    contributors to this software may be used to endorse or promote products 
  *    derived from this software without specific written permission.
  * 4. This software, including modifications and/or derivative works of this 
  *    software, must execute solely and exclusively on microcontroller or
  *    microprocessor devices manufactured by or for STMicroelectronics.
  * 5. Redistribution and use of this software other than as permitted under 
  *    this license is void and will automatically terminate your rights under 
  *    this license. 
  *
  * THIS SOFTWARE IS PROVIDED BY STMICROELECTRONICS AND CONTRIBUTORS "AS IS" 
  * AND ANY EXPRESS, IMPLIED OR STATUTORY WARRANTIES, INCLUDING, BUT NOT 
  * LIMITED TO, THE IMPLIED WARRANTIES OF MERCHANTABILITY, FITNESS FOR A 
  * PARTICULAR PURPOSE AND NON-INFRINGEMENT OF THIRD PARTY INTELLECTUAL PROPERTY
  * RIGHTS ARE DISCLAIMED TO THE FULLEST EXTENT PERMITTED BY LAW. IN NO EVENT 
  * SHALL STMICROELECTRONICS OR CONTRIBUTORS BE LIABLE FOR ANY DIRECT, INDIRECT,
  * INCIDENTAL, SPECIAL, EXEMPLARY, OR CONSEQUENTIAL DAMAGES (INCLUDING, BUT NOT
  * LIMITED TO, PROCUREMENT OF SUBSTITUTE GOODS OR SERVICES; LOSS OF USE, DATA, 
  * OR PROFITS; OR BUSINESS INTERRUPTION) HOWEVER CAUSED AND ON ANY THEORY OF 
  * LIABILITY, WHETHER IN CONTRACT, STRICT LIABILITY, OR TORT (INCLUDING 
  * NEGLIGENCE OR OTHERWISE) ARISING IN ANY WAY OUT OF THE USE OF THIS SOFTWARE,
  * EVEN IF ADVISED OF THE POSSIBILITY OF SUCH DAMAGE.
  *
  ************************************************
  */

/* Includes --------------------------------------*/
#include "main.h"
#include "stm32f4xx_hal.h"
#include "usb_host.h"
/* USER CODE BEGIN Includes */

#include "util.h"
#include "ADS1299.h"
#include "stdbool.h"
#include <stdlib.h>

// Includes y defines relacionados con el filtro FIR
#include "arm_math.h"
#include "math_helper.h"

/* USER CODE END Includes */

/* Private variables ------------------------------*/
ADC_HandleTypeDef hadc1;

SPI_HandleTypeDef hspi1;
SPI_HandleTypeDef hspi2;

UART_HandleTypeDef huart4;

/* USER CODE BEGIN PV */
/* Private variables -------------------------------*/

/* USER CODE END PV */

/* Private function prototypes ---------------------*/
void SystemClock_Config(void);
static void MX_GPIO_Init(void);
static void MX_SPI1_Init(void);
static void MX_SPI2_Init(void);
static void MX_ADC1_Init(void);
static void MX_UART4_Init(void);
void MX_USB_HOST_Process(void);

/* USER CODE BEGIN PFP */
/* Private function prototypes ---------------------*/

	uint8_t commandFromESP (uint8_t command, SPI_HandleTypeDef *SPI);
	void float2ESP (float32_t data2ESP, SPI_HandleTypeDef *SPI);
	void floatArray2ESP ( float32_t data2ESP_array[], int N_elementos, SPI_HandleTypeDef *SPI, UART_HandleTypeDef *huart4);

/* USER CODE END PFP */

/* USER CODE BEGIN 0 */

// global variables
unsigned long blink_interval_millis;
float32_t channel_1 [LENGTH_SAMPLES];
float32_t channel_2 [LENGTH_SAMPLES];
float32_t channel_3 [LENGTH_SAMPLES];
float32_t channel_4 [LENGTH_SAMPLES];
float32_t channel_5 [LENGTH_SAMPLES];
float32_t channel_6 [LENGTH_SAMPLES];
float32_t channel_7 [LENGTH_SAMPLES];
float32_t channel_8 [LENGTH_SAMPLES];

bool wait = false;
uint8_t bucle=1;
uint8_t k = 0;
int icanal=1;
int i;
bool acabar = false;
bool Data_ready = true;
uint8_t comando_temp = 0x00;

uint8_t data[27];

uint8_t command = 0x00;

int channel = 1;

float32_t data2ESP = 1234.5678;

bool shared_negative_electrode = true;

/* -------------------------------------------------
 * Declare Test output buffer
 * -------------------------------------------------*/
static float32_t testOutput[LENGTH_SAMPLES];
/* -------------------------------------------------
 * Declare State buffer of size (numTaps + blockSize - 1)
 * -------------------------------------------------*/
static float32_t firStateF32[BLOCK_SIZE + NUM_TAPS - 1];
/* -------------------------------------------------
** FIR Coefficients buffer generated using fir1() MATLAB function.
** fir1(28, 6/24)
** -------------------------------------------------*/

const float32_t firCoeffs32[NUM_TAPS] = {-0.0138029831773458, 0.0395399929599544, 0.0428349923732840, -0.0175674331348036, -0.0605461382912526, -0.0197781448960783, 0.0543574323501128, 0.0566917462875556, -0.0224447398190889, -0.0748391427030912, -0.0236934079009084, 0.0631991587299459, 0.0640421490722732, -0.0246563468927641, 0.920000000000000, -0.0246563468927641, 0.0640421490722732, 0.0631991587299459, -0.0236934079009084, -0.0748391427030912, -0.0224447398190889, 0.0566917462875556, 0.0543574323501128, -0.0197781448960783, -0.0605461382912526, -0.0175674331348036, 0.0428349923732840, 0.0395399929599544, -0.0138029831773458};

/* ------------------------------------------------------------------
 * Global variables for FIR LPF Example
 * ------------------------------------------------------------------- */
uint32_t blockSize = BLOCK_SIZE;
uint32_t numBlocks = LENGTH_SAMPLES/BLOCK_SIZE;
float32_t  snr;
	
// Inicialización de todas las variables relacionadas con el filtrado
	int i;
  arm_fir_instance_f32 S;
  arm_status status;
  float32_t  *inputF32 	= &channel_1[0];			// Definición del puntero de la señal a filtrar.
	float32_t  *outputF32 = &testOutput[0];			// Definición del puntero de la señal filtrada.

/* USER CODE END 0 */

int main(void)
{

  /* USER CODE BEGIN 1 */
	
  /* USER CODE END 1 */

  /* MCU Configuration----------------------------------------------------------*/

  /* Reset of all peripherals, Initializes the Flash interface and the Systick. */
  HAL_Init();

  /* USER CODE BEGIN Init */

	
  /* USER CODE END Init */

  /* Configure the system clock */
  SystemClock_Config();

  /* USER CODE BEGIN SysInit */
	
		
  /* USER CODE END SysInit */

  /* Initialize all configured peripherals */
  MX_GPIO_Init();
  MX_SPI1_Init();
  MX_SPI2_Init();
  MX_ADC1_Init();
  MX_UART4_Init();
  MX_USB_HOST_Init();

  /* USER CODE BEGIN 2 */
	
	//Inicialización del estado de ciertos pines
	
	HAL_GPIO_WritePin(DRDY_N_GPIO_Port,DRDY_N_Pin, GPIO_PIN_SET);
	HAL_GPIO_WritePin(LED_GPIO_Port,LED_Pin, GPIO_PIN_SET);
	HAL_GPIO_WritePin(A_START_GPIO_Port, A_START_Pin, GPIO_PIN_SET);

	// Reiniciamos el ESP12-E y lo dejamos habilitado
	
	HAL_GPIO_WritePin(EN_WIFI_GPIO_Port, EN_WIFI_Pin, GPIO_PIN_RESET);
	HAL_Delay(100);
	HAL_GPIO_WritePin(EN_WIFI_GPIO_Port, EN_WIFI_Pin, GPIO_PIN_SET);

	// Parpadeo del LED del ADS y el STM para comprobar si han arrancado bien
	
	HAL_GPIO_WritePin(LED_GPIO_Port,LED_Pin, GPIO_PIN_RESET); 	//LED ON
	adc_wreg(GPIO, 0x1C, &hspi2);				// Led 1 on, led 2 off
	HAL_Delay(500);
	HAL_GPIO_WritePin(LED_GPIO_Port,LED_Pin, GPIO_PIN_SET); 	//LED OFF
	adc_wreg(GPIO, 0x2C, &hspi2);				// Led 1 on, led 2 off
	HAL_Delay(500);
	
	// Reiniciamos el ADS y lo dejamos habilitado

	HAL_GPIO_WritePin(A_RESET_N_GPIO_Port, A_RESET_N_Pin, GPIO_PIN_RESET);
	HAL_Delay(100);
	HAL_GPIO_WritePin(A_RESET_N_GPIO_Port, A_RESET_N_Pin, GPIO_PIN_SET);
	
	uint8_t RESET_opcode = 0x06;
	uint8_t zero = 0x00;
	
	HAL_GPIO_WritePin(A_CS0_N_GPIO_Port, A_CS0_N_Pin, GPIO_PIN_RESET);
	HAL_SPI_TransmitReceive(&hspi2, &RESET_opcode, &zero, 1, 100);
	HAL_GPIO_WritePin(A_CS0_N_GPIO_Port, A_CS0_N_Pin, GPIO_PIN_SET);
	HAL_Delay(1000);
	
	// Initial configuration of the ADS
	uint8_t config [4];
	uint8_t config_channel [8];
	
	config[0] = CONFIG1_reserved | DR_250_SPS;
	config[1] = CONFIG2_reserved | INT_CAL | CAL_FREQ_SLOW;
	config[2] = PD_REFBUF | CONFIG3_reserved | BIASREF_INT | PD_BIAS;
	config[3] = PD_LOFF_COMP;
	
	config_channel [0] = TEST_SIGNAL | GAIN_24X;
	config_channel [1] = TEST_SIGNAL | GAIN_24X;
	config_channel [2] = TEST_SIGNAL | GAIN_24X;
	config_channel [3] = TEST_SIGNAL | GAIN_24X;
	config_channel [4] = TEST_SIGNAL | GAIN_24X;
	config_channel [5] = TEST_SIGNAL | GAIN_24X;
	config_channel [6] = TEST_SIGNAL | GAIN_24X;
	config_channel [7] = TEST_SIGNAL | GAIN_24X;
	
	configADS(config, config_channel, &hspi2, &huart4);
	
	uint8_t gain[8] = {0, 0, 0, 0, 0, 0, 0, 0};
	for (int i = 0; i<8; i++)
	{
		gain[i] = calcular_ganancia(config_channel[i]);
	}
	
	//==============================================================



  /* USER CODE END 2 */

  /* Infinite loop */
  /* USER CODE BEGIN WHILE */
  while (1)
  {
	uint8_t command2ESP = 0x02;
		// Begin of the states machine
		
		// Wait for command from ESP
	while (command == 0x00)
	{
		command = commandFromESP(command2ESP, &hspi1);
	}
	
	switch (command){
		case (0xF0):		// Cambio en la configuración
			for (int i=0; i<4; i++){
				config[i] = commandFromESP(command, &hspi1);
			}
			for (int i=0; i<8; i++){
				config_channel[i] = commandFromESP(command, &hspi1);
			}
			configADS(config, config_channel, &hspi2, &huart4);
			
			for (int i = 0; i<8; i++)
			{
				gain[i] = calcular_ganancia(config_channel[i]);
			}
			
			command = 0x00;
			
		break;
			
		case (0x01):	// Lectura de un dato y envío al ESP
		
			adquire_single_data(data, &hspi2, &huart4);
			
//			one_shot(data, &hspi2, &huart4);
		
			//adc_wreg(GPIO, 0x2C, &hspi2);				// Led 1 on, led 2 off	
			
			for (int i = 1; i<=8; i++)
			{					
			data2ESP = byte2float(data[i*3], data[i*3 +1], data[i*3 + 2], gain[i]);
			float2ESP (data2ESP, &hspi1);
			}
			//adc_wreg(GPIO, 0x1C,&hspi2);				// Led 1 off, led 2 on			
			
			Serial_println_N(data[3], &huart4);
			Serial_println_N(data[4], &huart4);
			Serial_println_N(data[5], &huart4);
			
		command = 0x00;
				
		break;
		
		case (0x02):	// Lectura continua de datos y envío al ESP
			
		adquire_array_data (data, channel_1, channel_2, channel_3, channel_4, channel_5, channel_6, channel_7, channel_8, gain, &hspi2, &huart4);

	// ----------------------------------------------------------------------
  // Call the FIR process function for every blockSize samples
  // ------------------------------------------------------------------- 
  
		for(i=0; i < (BLOCK_SIZE + NUM_TAPS - 1); i++) 
		{
			firStateF32[i]=channel_1[i];
		}
		// Call FIR init function to initialize the instance structure. 
		arm_fir_init_f32(&S, NUM_TAPS, (float32_t *)&firCoeffs32[0], &firStateF32[0], blockSize);
		
		for(i=0; i < numBlocks; i++) //
		{
			arm_fir_f32(&S, inputF32 + (i * blockSize), outputF32 + (i * blockSize), blockSize);
		}
		
//			one_shot_array (data, channel_1, 1, &hspi2, &huart4);
			
			floatArray2ESP (testOutput, LENGTH_SAMPLES, &hspi1, &huart4);
			//floatArray2ESP (channel_1, LENGTH_SAMPLES, &hspi1, &huart4);
			command = 0x00;
				
		break;
			
		default:
			
				command = 0x00;
		
		break;
	}
	
		
  /* USER CODE END WHILE */
    MX_USB_HOST_Process();

  /* USER CODE BEGIN 3 */

  }
  /* USER CODE END 3 */

}

/** System Clock Configuration
*/
void SystemClock_Config(void)
{

  RCC_OscInitTypeDef RCC_OscInitStruct;
  RCC_ClkInitTypeDef RCC_ClkInitStruct;

    /**Configure the main internal regulator output voltage 
    */
  __HAL_RCC_PWR_CLK_ENABLE();

  __HAL_PWR_VOLTAGESCALING_CONFIG(PWR_REGULATOR_VOLTAGE_SCALE1);

    /**Initializes the CPU, AHB and APB busses clocks 
    */
  RCC_OscInitStruct.OscillatorType = RCC_OSCILLATORTYPE_HSE;
  RCC_OscInitStruct.HSEState = RCC_HSE_ON;
  RCC_OscInitStruct.PLL.PLLState = RCC_PLL_ON;
  RCC_OscInitStruct.PLL.PLLSource = RCC_PLLSOURCE_HSE;
  RCC_OscInitStruct.PLL.PLLM = 8;
  RCC_OscInitStruct.PLL.PLLN = 336;
  RCC_OscInitStruct.PLL.PLLP = RCC_PLLP_DIV2;
  RCC_OscInitStruct.PLL.PLLQ = 7;
  if (HAL_RCC_OscConfig(&RCC_OscInitStruct) != HAL_OK)
  {
    _Error_Handler(__FILE__, __LINE__);
  }

    /**Initializes the CPU, AHB and APB busses clocks 
    */
  RCC_ClkInitStruct.ClockType = RCC_CLOCKTYPE_HCLK|RCC_CLOCKTYPE_SYSCLK
                              |RCC_CLOCKTYPE_PCLK1|RCC_CLOCKTYPE_PCLK2;
  RCC_ClkInitStruct.SYSCLKSource = RCC_SYSCLKSOURCE_PLLCLK;
  RCC_ClkInitStruct.AHBCLKDivider = RCC_SYSCLK_DIV1;
  RCC_ClkInitStruct.APB1CLKDivider = RCC_HCLK_DIV4;
  RCC_ClkInitStruct.APB2CLKDivider = RCC_HCLK_DIV2;

  if (HAL_RCC_ClockConfig(&RCC_ClkInitStruct, FLASH_LATENCY_5) != HAL_OK)
  {
    _Error_Handler(__FILE__, __LINE__);
  }

    /**Enables the Clock Security System 
    */
  HAL_RCC_EnableCSS();

    /**Configure the Systick interrupt time 
    */
  HAL_SYSTICK_Config(HAL_RCC_GetHCLKFreq()/1000);

    /**Configure the Systick 
    */
  HAL_SYSTICK_CLKSourceConfig(SYSTICK_CLKSOURCE_HCLK);

  /* SysTick_IRQn interrupt configuration */
  HAL_NVIC_SetPriority(SysTick_IRQn, 0, 0);
}

/* ADC1 init function */
static void MX_ADC1_Init(void)
{

  ADC_ChannelConfTypeDef sConfig;

    /**Configure the global features of the ADC (Clock, Resolution, Data Alignment and number of conversion) 
    */
  hadc1.Instance = ADC1;
  hadc1.Init.ClockPrescaler = ADC_CLOCK_SYNC_PCLK_DIV4;
  hadc1.Init.Resolution = ADC_RESOLUTION_12B;
  hadc1.Init.ScanConvMode = DISABLE;
  hadc1.Init.ContinuousConvMode = DISABLE;
  hadc1.Init.DiscontinuousConvMode = DISABLE;
  hadc1.Init.ExternalTrigConvEdge = ADC_EXTERNALTRIGCONVEDGE_NONE;
  hadc1.Init.ExternalTrigConv = ADC_SOFTWARE_START;
  hadc1.Init.DataAlign = ADC_DATAALIGN_RIGHT;
  hadc1.Init.NbrOfConversion = 1;
  hadc1.Init.DMAContinuousRequests = DISABLE;
  hadc1.Init.EOCSelection = ADC_EOC_SINGLE_CONV;
  if (HAL_ADC_Init(&hadc1) != HAL_OK)
  {
    _Error_Handler(__FILE__, __LINE__);
  }

    /**Configure for the selected ADC regular channel its corresponding rank in the sequencer and its sample time. 
    */
  sConfig.Channel = ADC_CHANNEL_1;
  sConfig.Rank = 1;
  sConfig.SamplingTime = ADC_SAMPLETIME_3CYCLES;
  if (HAL_ADC_ConfigChannel(&hadc1, &sConfig) != HAL_OK)
  {
    _Error_Handler(__FILE__, __LINE__);
  }

}

/* SPI1 init function */
static void MX_SPI1_Init(void)
{

  /* SPI1 parameter configuration*/
  hspi1.Instance = SPI1;
  hspi1.Init.Mode = SPI_MODE_SLAVE;
  hspi1.Init.Direction = SPI_DIRECTION_2LINES;
  hspi1.Init.DataSize = SPI_DATASIZE_8BIT;
  hspi1.Init.CLKPolarity = SPI_POLARITY_LOW;
  hspi1.Init.CLKPhase = SPI_PHASE_2EDGE;
  hspi1.Init.NSS = SPI_NSS_HARD_INPUT;
  hspi1.Init.FirstBit = SPI_FIRSTBIT_MSB;
  hspi1.Init.TIMode = SPI_TIMODE_DISABLE;
  hspi1.Init.CRCCalculation = SPI_CRCCALCULATION_DISABLE;
  hspi1.Init.CRCPolynomial = 10;
  if (HAL_SPI_Init(&hspi1) != HAL_OK)
  {
    _Error_Handler(__FILE__, __LINE__);
  }

}

/* SPI2 init function */
static void MX_SPI2_Init(void)
{

  /* SPI2 parameter configuration*/
  hspi2.Instance = SPI2;
  hspi2.Init.Mode = SPI_MODE_MASTER;
  hspi2.Init.Direction = SPI_DIRECTION_2LINES;
  hspi2.Init.DataSize = SPI_DATASIZE_8BIT;
  hspi2.Init.CLKPolarity = SPI_POLARITY_LOW;
  hspi2.Init.CLKPhase = SPI_PHASE_2EDGE;
  hspi2.Init.NSS = SPI_NSS_SOFT;
  hspi2.Init.BaudRatePrescaler = SPI_BAUDRATEPRESCALER_2;
  hspi2.Init.FirstBit = SPI_FIRSTBIT_MSB;
  hspi2.Init.TIMode = SPI_TIMODE_DISABLE;
  hspi2.Init.CRCCalculation = SPI_CRCCALCULATION_DISABLE;
  hspi2.Init.CRCPolynomial = 10;
  if (HAL_SPI_Init(&hspi2) != HAL_OK)
  {
    _Error_Handler(__FILE__, __LINE__);
  }

}

/* UART4 init function */
static void MX_UART4_Init(void)
{

  huart4.Instance = UART4;
  huart4.Init.BaudRate = 115200;
  huart4.Init.WordLength = UART_WORDLENGTH_8B;
  huart4.Init.StopBits = UART_STOPBITS_1;
  huart4.Init.Parity = UART_PARITY_NONE;
  huart4.Init.Mode = UART_MODE_TX_RX;
  huart4.Init.HwFlowCtl = UART_HWCONTROL_NONE;
  huart4.Init.OverSampling = UART_OVERSAMPLING_16;
  if (HAL_UART_Init(&huart4) != HAL_OK)
  {
    _Error_Handler(__FILE__, __LINE__);
  }

}

/** Configure pins as 
        * Analog 
        * Input 
        * Output
        * EVENT_OUT
        * EXTI
*/
static void MX_GPIO_Init(void)
{

  GPIO_InitTypeDef GPIO_InitStruct;

  /* GPIO Ports Clock Enable */
  __HAL_RCC_GPIOH_CLK_ENABLE();
  __HAL_RCC_GPIOC_CLK_ENABLE();
  __HAL_RCC_GPIOA_CLK_ENABLE();
  __HAL_RCC_GPIOB_CLK_ENABLE();

  /*Configure GPIO pin Output Level */
  HAL_GPIO_WritePin(GPIOC, EN_WIFI_Pin|A_START_Pin|A_RESET_N_Pin, GPIO_PIN_SET);

  /*Configure GPIO pin Output Level */
  HAL_GPIO_WritePin(EN_BT_GPIO_Port, EN_BT_Pin, GPIO_PIN_RESET);

  /*Configure GPIO pin Output Level */
  HAL_GPIO_WritePin(GPIOA, DRDY_N_Pin|LED_Pin, GPIO_PIN_SET);

  /*Configure GPIO pin Output Level */
  HAL_GPIO_WritePin(A_CS1_N_GPIO_Port, A_CS1_N_Pin, GPIO_PIN_RESET);

  /*Configure GPIO pin Output Level */
  HAL_GPIO_WritePin(A_CS0_N_GPIO_Port, A_CS0_N_Pin, GPIO_PIN_SET);

  /*Configure GPIO pin Output Level */
  HAL_GPIO_WritePin(EN_USB_GPIO_Port, EN_USB_Pin, GPIO_PIN_RESET);

  /*Configure GPIO pins : EN_WIFI_Pin EN_BT_Pin A_START_Pin A_RESET_N_Pin */
  GPIO_InitStruct.Pin = EN_WIFI_Pin|EN_BT_Pin|A_START_Pin|A_RESET_N_Pin;
  GPIO_InitStruct.Mode = GPIO_MODE_OUTPUT_PP;
  GPIO_InitStruct.Pull = GPIO_NOPULL;
  GPIO_InitStruct.Speed = GPIO_SPEED_FREQ_LOW;
  HAL_GPIO_Init(GPIOC, &GPIO_InitStruct);

  /*Configure GPIO pins : DRDY_N_Pin LED_Pin EN_USB_Pin */
  GPIO_InitStruct.Pin = DRDY_N_Pin|LED_Pin|EN_USB_Pin;
  GPIO_InitStruct.Mode = GPIO_MODE_OUTPUT_PP;
  GPIO_InitStruct.Pull = GPIO_NOPULL;
  GPIO_InitStruct.Speed = GPIO_SPEED_FREQ_LOW;
  HAL_GPIO_Init(GPIOA, &GPIO_InitStruct);

  /*Configure GPIO pins : START_Pin A_DRDY_N_Pin */
  GPIO_InitStruct.Pin = START_Pin|A_DRDY_N_Pin;
  GPIO_InitStruct.Mode = GPIO_MODE_INPUT;
  GPIO_InitStruct.Pull = GPIO_NOPULL;
  HAL_GPIO_Init(GPIOC, &GPIO_InitStruct);

  /*Configure GPIO pins : A_CS1_N_Pin A_CS0_N_Pin */
  GPIO_InitStruct.Pin = A_CS1_N_Pin|A_CS0_N_Pin;
  GPIO_InitStruct.Mode = GPIO_MODE_OUTPUT_PP;
  GPIO_InitStruct.Pull = GPIO_NOPULL;
  GPIO_InitStruct.Speed = GPIO_SPEED_FREQ_LOW;
  HAL_GPIO_Init(GPIOB, &GPIO_InitStruct);

}

/* USER CODE BEGIN 4 */

uint8_t commandFromESP(uint8_t command, SPI_HandleTypeDef *hspi1)
{
	uint8_t response = 0x00;	

		HAL_GPIO_WritePin(DRDY_N_GPIO_Port,DRDY_N_Pin, GPIO_PIN_RESET);
		HAL_SPI_TransmitReceive(hspi1, &command, &response, 1, 100);
		HAL_GPIO_WritePin(DRDY_N_GPIO_Port,DRDY_N_Pin, GPIO_PIN_SET);
		HAL_GPIO_TogglePin(LED_GPIO_Port,LED_Pin); 	//LED OFF
		HAL_Delay(50);
	
	return response;
}

void float2ESP (float32_t data2ESP, SPI_HandleTypeDef *SPI)
{	
			union miDato{
				struct
				{
					uint8_t  b[4];     // Array de bytes de tamaño igual al tamaño de la primera variable: int = 2 bytes, float = 4 bytes
				}split;
					float32_t fval;
			 } float_data; 

			 float_data.fval=data2ESP;
			 
			 HAL_GPIO_WritePin(DRDY_N_GPIO_Port,DRDY_N_Pin, GPIO_PIN_RESET);
			 HAL_SPI_Transmit(SPI, float_data.split.b, 4, 100);
			 HAL_GPIO_WritePin(DRDY_N_GPIO_Port,DRDY_N_Pin, GPIO_PIN_SET);
			 HAL_GPIO_TogglePin(LED_GPIO_Port,LED_Pin); 	//LED OFF
			 HAL_Delay(10);
}

void floatArray2ESP ( float32_t data2ESP_array[], int N_elementos, SPI_HandleTypeDef *SPI, UART_HandleTypeDef *huart4)
{
	union miDato{
		struct
		{
			uint8_t  b[4];     // Array de bytes de tamaño igual al tamaño de la primera variable: int = 2 bytes, float = 4 bytes
		}split;
			float32_t fval;
	 } float_data; 
	
	uint8_t buffer [1000]; //tamaño máximo del buffer a usar
	int i = 0;
	 
	for(i = 0; i<N_elementos; i++)
	 {
		 float_data.fval 	= data2ESP_array[i];
		 buffer[i*4] 	 		= float_data.split.b[0];
		 buffer[i*4+1] 		= float_data.split.b[1];
		 buffer[i*4+2] 		= float_data.split.b[2];
		 buffer[i*4+3] 		= float_data.split.b[3];
	 }
	 
	 HAL_GPIO_WritePin(DRDY_N_GPIO_Port,DRDY_N_Pin, GPIO_PIN_RESET);
	 HAL_SPI_Transmit(SPI, buffer, 1000, 50);
	 HAL_GPIO_WritePin(DRDY_N_GPIO_Port,DRDY_N_Pin, GPIO_PIN_SET);
	 HAL_GPIO_TogglePin(LED_GPIO_Port,LED_Pin); 	//LED OFF

}

/* USER CODE END 4 */

/**
  * @brief  This function is executed in case of error occurrence.
  * @param  None
  * @retval None
  */
void _Error_Handler(char * file, int line)
{
  /* USER CODE BEGIN Error_Handler_Debug */
  /* User can add his own implementation to report the HAL error return state */
  while(1) 
  {
  }
  /* USER CODE END Error_Handler_Debug */ 
}

#ifdef USE_FULL_ASSERT

/**
   * @brief Reports the name of the source file and the source line number
   * where the assert_param error has occurred.
   * @param file: pointer to the source file name
   * @param line: assert_param error line source number
   * @retval None
   */
void assert_failed(uint8_t* file, uint32_t line)
{
  /* USER CODE BEGIN 6 */
  /* User can add his own implementation to report the file name and line number,
    ex: printf("Wrong parameters value: file %s on line %d\r\n", file, line) */
  /* USER CODE END 6 */

}

#endif

/**
  * @}
  */ 

/**
  * @}
*/ 

/************************ (C) COPYRIGHT STMicroelectronics *****END OF FILE****/

\end{lstlisting}

\begin{lstlisting}[label=algoritmo:STM32F4:main.h,style = STM-code,frame=single,caption=STM32F4:main.h]
/**
  ******************************************************
  * File Name          : main.hpp
  * Description        : This file contains the common defines of the application
  ******************************************************
  * This notice applies to any and all portions of this file
  * that are not between comment pairs USER CODE BEGIN and
  * USER CODE END. Other portions of this file, whether 
  * inserted by the user or by software development tools
  * are owned by their respective copyright owners.
  *
  * Copyright (c) 2018 STMicroelectronics International N.V. 
  * All rights reserved.
  *
  * Redistribution and use in source and binary forms, with or without 
  * modification, are permitted, provided that the following conditions are met:
  *
  * 1. Redistribution of source code must retain the above copyright notice, 
  *    this list of conditions and the following disclaimer.
  * 2. Redistributions in binary form must reproduce the above copyright notice,
  *    this list of conditions and the following disclaimer in the documentation
  *    and/or other materials provided with the distribution.
  * 3. Neither the name of STMicroelectronics nor the names of other 
  *    contributors to this software may be used to endorse or promote products 
  *    derived from this software without specific written permission.
  * 4. This software, including modifications and/or derivative works of this 
  *    software, must execute solely and exclusively on microcontroller or
  *    microprocessor devices manufactured by or for STMicroelectronics.
  * 5. Redistribution and use of this software other than as permitted under 
  *    this license is void and will automatically terminate your rights under 
  *    this license. 
  *
  * THIS SOFTWARE IS PROVIDED BY STMICROELECTRONICS AND CONTRIBUTORS "AS IS" 
  * AND ANY EXPRESS, IMPLIED OR STATUTORY WARRANTIES, INCLUDING, BUT NOT 
  * LIMITED TO, THE IMPLIED WARRANTIES OF MERCHANTABILITY, FITNESS FOR A 
  * PARTICULAR PURPOSE AND NON-INFRINGEMENT OF THIRD PARTY INTELLECTUAL PROPERTY
  * RIGHTS ARE DISCLAIMED TO THE FULLEST EXTENT PERMITTED BY LAW. IN NO EVENT 
  * SHALL STMICROELECTRONICS OR CONTRIBUTORS BE LIABLE FOR ANY DIRECT, INDIRECT,
  * INCIDENTAL, SPECIAL, EXEMPLARY, OR CONSEQUENTIAL DAMAGES (INCLUDING, BUT NOT
  * LIMITED TO, PROCUREMENT OF SUBSTITUTE GOODS OR SERVICES; LOSS OF USE, DATA, 
  * OR PROFITS; OR BUSINESS INTERRUPTION) HOWEVER CAUSED AND ON ANY THEORY OF 
  * LIABILITY, WHETHER IN CONTRACT, STRICT LIABILITY, OR TORT (INCLUDING 
  * NEGLIGENCE OR OTHERWISE) ARISING IN ANY WAY OUT OF THE USE OF THIS SOFTWARE,
  * EVEN IF ADVISED OF THE POSSIBILITY OF SUCH DAMAGE.
  *
  ***************************************************
  */
/* Define to prevent recursive inclusion -------------------------------------*/
#ifndef __MAIN_H
#define __MAIN_H
  /* Includes ------------------------------------------------------------------*/

/* Includes ------------------------------------------------------------------*/
/* USER CODE BEGIN Includes */

/* USER CODE END Includes */

/* Private define ------------------------------------------------------------*/

#define EN_WIFI_Pin GPIO_PIN_0
#define EN_WIFI_GPIO_Port GPIOC
#define EN_BT_Pin GPIO_PIN_1
#define EN_BT_GPIO_Port GPIOC
#define DRDY_N_Pin GPIO_PIN_2
#define DRDY_N_GPIO_Port GPIOA
#define START_Pin GPIO_PIN_4
#define START_GPIO_Port GPIOC
#define A_CS1_N_Pin GPIO_PIN_11
#define A_CS1_N_GPIO_Port GPIOB
#define A_CS0_N_Pin GPIO_PIN_12
#define A_CS0_N_GPIO_Port GPIOB
#define A_SCK_Pin GPIO_PIN_13
#define A_SCK_GPIO_Port GPIOB
#define A_MISO_Pin GPIO_PIN_14
#define A_MISO_GPIO_Port GPIOB
#define A_MOSI_Pin GPIO_PIN_15
#define A_MOSI_GPIO_Port GPIOB
#define A_DRDY_N_Pin GPIO_PIN_6
#define A_DRDY_N_GPIO_Port GPIOC
#define A_START_Pin GPIO_PIN_7
#define A_START_GPIO_Port GPIOC
#define A_RESET_N_Pin GPIO_PIN_8
#define A_RESET_N_GPIO_Port GPIOC
#define LED_Pin GPIO_PIN_9
#define LED_GPIO_Port GPIOA
#define EN_USB_Pin GPIO_PIN_10
#define EN_USB_GPIO_Port GPIOA

/* ################# Assert Selection ############### */
/**
  * @brief Uncomment the line below to expanse the "assert_param" macro in the 
  *        HAL drivers code
  */
/* #define USE_FULL_ASSERT    1U */

/* USER CODE BEGIN Private defines */

#define BLINK_INTERVAL_SETUP 100;
#define BLINK_INTERVAL_WAITING 500;
#define BLINK_INTERVAL_SENDING 2000;

// Defines relacionados con el filtro FIR
#define LENGTH_SAMPLES  250
#define SNR_THRESHOLD_F32    140.0f
#define BLOCK_SIZE            32
#define NUM_TAPS              29

/* USER CODE END Private defines */

#ifdef __cplusplus
 extern "C" {
#endif
void _Error_Handler(char *, int);

#define Error_Handler() _Error_Handler(__FILE__, __LINE__)
#ifdef __cplusplus
}
#endif

/**
  * @}
  */ 

/**
  * @}
*/ 

#endif /* __MAIN_H */
/************************ (C) COPYRIGHT STMicroelectronics *****END OF FILE****/

\end{lstlisting}


\begin{lstlisting}[label=algoritmo:STM32F4:ADS1299.c,style = STM-code,frame=single,caption=STM32F4:ADS1299.c]
/**
  ***************************************************
  * File Name          : ADS1299.c
  * Description        : Funtion to interact with the ADS1299
  ***************************************************
  */

/* Includes ------------------------------------------------------------------*/

#include "ADS1299.h"
#include "main.h"		
#include "util.h"
#include "stdbool.h"
#include <stdlib.h>

void Blinky(SPI_HandleTypeDef *hspi2)
{
	
		uint8_t opcode_1 = WREG | GPIO; //0x54  (WREG + GPIO) Registro encargado de escribir la configuración de los GPIO
		uint8_t opcode_2 = 0x00;		//Registros a leer (N-1)
		uint8_t reg_leido = 0x00;
		uint8_t reg_LED_ON 	= 0x00;
		uint8_t reg_LED_OFF 	= 0xF0;
		
	
		HAL_GPIO_WritePin(A_CS0_N_GPIO_Port, A_CS0_N_Pin, GPIO_PIN_RESET);
		HAL_SPI_TransmitReceive(hspi2, &opcode_1, &reg_leido, 1, HAL_MAX_DELAY);
		HAL_SPI_TransmitReceive(hspi2, &opcode_2, &reg_leido, 1, HAL_MAX_DELAY);					
		HAL_SPI_TransmitReceive(hspi2, &reg_LED_ON, &reg_leido, 1, HAL_MAX_DELAY); //LED ON
		HAL_GPIO_WritePin(A_CS0_N_GPIO_Port, A_CS0_N_Pin, GPIO_PIN_SET);
	
		HAL_Delay(500);
	
		HAL_GPIO_WritePin(A_CS0_N_GPIO_Port, A_CS0_N_Pin, GPIO_PIN_RESET);
		HAL_SPI_TransmitReceive(hspi2, &opcode_1, &reg_leido, 1, HAL_MAX_DELAY);
		HAL_SPI_TransmitReceive(hspi2, &opcode_2, &reg_leido, 1, HAL_MAX_DELAY);					
		HAL_SPI_TransmitReceive(hspi2, &reg_LED_OFF, &reg_leido, 1, HAL_MAX_DELAY); //LED OFF
		HAL_GPIO_WritePin(A_CS0_N_GPIO_Port, A_CS0_N_Pin, GPIO_PIN_SET);
		
		HAL_Delay(500);
}

void adc_send_command(uint8_t cmd, SPI_HandleTypeDef *SPI)
{
	uint8_t comando_temp = cmd;
	uint8_t zero = 0x00;
	//IPIN_MASTER_CS:
	HAL_GPIO_WritePin(A_CS0_N_GPIO_Port, A_CS0_N_Pin, GPIO_PIN_RESET);
	HAL_SPI_TransmitReceive(SPI, &comando_temp, &zero, 1, 100); //SPI.transfer(cmd);
	HAL_Delay(1);
	HAL_GPIO_WritePin(A_CS0_N_GPIO_Port, A_CS0_N_Pin, GPIO_PIN_SET);
}

uint8_t adc_rreg(uint8_t reg, SPI_HandleTypeDef *SPI)
{
	uint8_t val = 0x00;
	uint8_t zero_t = 0x00;
	uint8_t zero_r = 0x00;
	uint8_t temp = RREG | reg;
	
	HAL_GPIO_WritePin(A_CS0_N_GPIO_Port, A_CS0_N_Pin, GPIO_PIN_RESET);
	HAL_SPI_TransmitReceive(SPI, &temp, &zero_r, 1, 100);
	HAL_SPI_TransmitReceive(SPI, &zero_t, &zero_r, 1, 100);	// number of registers to be read/written
	HAL_SPI_TransmitReceive(SPI, &zero_t, &val, 1, 100);

	HAL_Delay(1);
	HAL_GPIO_WritePin(A_CS0_N_GPIO_Port, A_CS0_N_Pin, GPIO_PIN_SET);

	return val;
}

void adc_wreg(uint8_t reg, uint8_t val, SPI_HandleTypeDef *SPI)
{
	uint8_t zero_t = 0x00;
	uint8_t zero_r = 0x00;
	uint8_t reg_temp = WREG | reg;
	uint8_t val_temp = val;
	//uint8_t buffer [3] = {WREG | reg, 0x00, val};
	
	// IPIN_MASTER_CS
	HAL_GPIO_WritePin(A_CS0_N_GPIO_Port, A_CS0_N_Pin, GPIO_PIN_RESET);
	// ADS1298::WREG
	//HAL_SPI_Transmit(SPI, buffer, 3, 100);
	HAL_SPI_TransmitReceive(SPI, &reg_temp, &zero_r, 1, 100);
	HAL_SPI_TransmitReceive(SPI, &zero_t, &zero_r, 1, 100);	// number of registers to be read/written
	HAL_SPI_TransmitReceive(SPI, &val_temp, &zero_r, 1, 100);	

	HAL_Delay(1);
	HAL_GPIO_WritePin(A_CS0_N_GPIO_Port, A_CS0_N_Pin, GPIO_PIN_SET);
}

void read_data_frame(uint8_t data [], SPI_HandleTypeDef *SPI)
{
	uint8_t zero [27] = {0,0,0,0,0,0,0,0,0,0,0,0,0,0,0,0,0,0,0,0,0,0,0,0,0,0,0};
	
	// IPIN_MASTER_CS	
//		HAL_GPIO_WritePin(A_CS0_N_GPIO_Port, A_CS0_N_Pin, GPIO_PIN_RESET);
//		HAL_SPI_Receive(SPI, data, 27, 100);
//			for(int i = 0; i<27; i++)
//			{
				HAL_SPI_TransmitReceive(SPI, zero, data, 27, 100);
//			}
		//HAL_Delay(1);	// is this needed? Yes, it requires a 4Tclk = 0.2ms wait.
	
//		HAL_GPIO_WritePin(A_CS0_N_GPIO_Port, A_CS0_N_Pin, GPIO_PIN_SET);
//		while (HAL_GPIO_ReadPin(A_DRDY_N_GPIO_Port, A_DRDY_N_Pin) == GPIO_PIN_RESET) {}
}

void update_bias_ref(uint8_t data[], SPI_HandleTypeDef *SPI)
{

	static uint8_t last_loff_statp = 0xFF;
	static uint8_t last_loff_statn = 0xFF;
	static unsigned samples_since_last_bias_change = 0;
	const unsigned min_samples_between_bias_changes = 100;

	uint8_t loff_statp = frame_loff_statp(data);
	uint8_t leads_on_p = ~loff_statp;
	uint8_t loff_statn = frame_loff_statn(data);
	uint8_t leads_on_n = ~loff_statn;

		bool shared_negative_electrode = true;
	
	if (shared_negative_electrode) {
		loff_statn |= 0x01;	// count only the single shared electrode
	}
	// if the lead-off status has changed...
	if (samples_since_last_bias_change >= min_samples_between_bias_changes
	    && (last_loff_statp != loff_statp
		|| last_loff_statn != loff_statn)) {

		// Send SDATAC Command (Stop Read Data Continuously mode)
		// TODO: starting and stopping the data collection like this will
		// create a glitch in all channels of the recording whenever the
		// leadoff status of any channel changes.  This could be fixed by
		// capturing all data in single-shot mode, triggered by an interrupt.
		adc_send_command(SDATAC, SPI);

		// Use only the leads that are connected to drive the bias electrode.
		adc_wreg(RLD_SENSP, leads_on_p, SPI);
		adc_wreg(RLD_SENSN, leads_on_n, SPI);

		// Put the Device Back in Read DATA Continuous Mode
		adc_send_command(RDATAC, SPI);

		last_loff_statp = loff_statp;
		last_loff_statn = loff_statn;
		samples_since_last_bias_change = 0;
	} else {
		++samples_since_last_bias_change;
	}
}


	//#ifdef __cplusplus
			uint8_t frame_loff_statp(uint8_t data[])
			{
				return ((data[0] << 4) | (data[1] >> 4));
			};
			uint8_t frame_loff_statn(uint8_t data[])
			{
				return ((data[1] << 4) | (data[2] >> 4));
			};
			uint8_t frame_loff_statp_i(uint8_t data[], int i)
				{
				return ((frame_loff_statp(data) >> i) & 1);
			};
			uint8_t frame_loff_statn_i(uint8_t data[], int i)
			{
				return ((frame_loff_statn(data) >> i) & 1);
			};
	//#endif
		//};

float32_t byte2float (uint8_t data_23_16, uint8_t data_15_8, uint8_t data_7_0, uint8_t ganancia)
	{
		float32_t value = 0;
		
		union miDato{
		struct
		{
			uint8_t  b[4];     // Array de bytes de tamaño igual al tamaño de la primera variable: int = 2 bytes, float = 4 bytes
		}split;
			long dato;
	 } long_data; 
		
	 long_data.dato = 0;
	 
		if (data_23_16>=0x80)
		{
			long_data.split.b[2] = ~data_23_16;
			long_data.split.b[1] = ~data_15_8;
			long_data.split.b[0] = ~data_7_0;
			value = long_data.dato;
			value = -value;
		}
		else
		{
			long_data.split.b[2] = data_23_16;
			long_data.split.b[1] = data_15_8;
			long_data.split.b[0] = data_7_0;
			value = long_data.dato;
		}
		
		value = value*4.5f/(8388607.0f*ganancia);
		
		return value;
	}
	
	
	void configADS(uint8_t config[], uint8_t config_channel[], SPI_HandleTypeDef *SPI, UART_HandleTypeDef *huart4)
	{
		int i;
		bool continuo = false;

		HAL_GPIO_WritePin(A_RESET_N_GPIO_Port, A_RESET_N_Pin, GPIO_PIN_RESET);
		HAL_Delay(100);
		HAL_GPIO_WritePin(A_RESET_N_GPIO_Port, A_RESET_N_Pin, GPIO_PIN_SET);
		
		for (i = 0; i <8; ++i) {    
			HAL_Delay(50);
		}

		// Send SDATAC Command (Stop Read Data Continuously mode)
		adc_send_command(SDATAC, SPI);

		// Power up the internal reference and wait for it to settle
			adc_wreg(CONFIG3, config[3-1], SPI);
		//adc_wreg(CONFIG3, PD_REFBUF | CONFIG3_reserved | BIASREF_INT | PD_BIAS, SPI); //Default mode
		//adc_wreg(CONFIG3, SPI); // 0xEC = 1110 1100	
		
		HAL_Delay(150);
		
		adc_wreg(CONFIG4, config[4-1], SPI);
		//adc_wreg(CONFIG4, PD_LOFF_COMP, SPI); // Default mode
		//adc_wreg(CONFIG4, 0x02, SPI); // 0x02 = 0000 0010
		
		adc_wreg(LOFF, COMP_TH_80 | ILEAD_OFF_12nA, SPI);
		adc_wreg(LOFF_SENSP, 0xFF, SPI);
		adc_wreg(LOFF_SENSN, 0x01, SPI);
		
		// Use lead-off sensing in all channels (but only drive one of the
		// negative leads if all of them are connected to one electrode)
		
		adc_wreg(CONFIG1, config[1-1], SPI);	// 250 SPS
		//adc_wreg(CONFIG1, CONFIG1_reserved | DR_250_SPS, SPI);	// 250 SPS - Default mode - 0x96
		
		adc_wreg(CONFIG2, config[2-1], SPI);
		//adc_wreg(CONFIG2, CONFIG2_reserved | INT_CAL | CAL_FREQ_SLOW, SPI); //| TEST_AMP | TEST_FREQ0);	// generate internal test signals - Default mode
		//adc_wreg(CONFIG2, 0xD0, SPI);  //D0 = 1101 0000
		
		// If we want to share a single negative electrode, tie the negative
		// inputs together using the BIAS_IN line.
		//uint8_t mux = RLD_DRN;

		// connect the negative channel to the (shared) BIAS_IN line
		// Set the first LIVE_CHANNELS_NUM channels to input signal
		for (i = 1; i <= LIVE_CHANNELS_NUM; ++i) {
			//adc_wreg(CHnSET + i, mux | GAIN_12X);
			//adc_wreg(CHnSET + i, ELECTRODE_INPUT | GAIN_12X, SPI); 
			adc_wreg(CHnSET + i, config_channel[i-1], SPI);
		}
		// Set all remaining channels to shorted inputs
		for (; i <= 8; ++i) {
			adc_wreg(CHnSET + i, SHORTED | PDn, SPI);
		}
		
		HAL_Delay(3 * 1000);
		
		HAL_GPIO_WritePin(A_START_GPIO_Port, A_START_Pin, GPIO_PIN_SET);
		
		//adc_send_command(START, SPI);
		if (continuo == true)
		{
			adc_send_command(RDATAC, SPI);
		}else
		{
			adc_send_command(SDATAC, SPI);
		}
		
	}
	
	void adquire_single_data (uint8_t data[], SPI_HandleTypeDef *SPI, UART_HandleTypeDef *huart4)
	{
	// IPIN_MASTER_CS	

			adc_send_command(RDATAC, SPI);
		
			while (HAL_GPIO_ReadPin(A_DRDY_N_GPIO_Port, A_DRDY_N_Pin) == GPIO_PIN_SET){}
		
			HAL_GPIO_WritePin(A_CS0_N_GPIO_Port, A_CS0_N_Pin, GPIO_PIN_RESET);
				
			read_data_frame(data, SPI);
			
			while (HAL_GPIO_ReadPin(A_DRDY_N_GPIO_Port, A_DRDY_N_Pin) == GPIO_PIN_RESET) {}
				
			HAL_GPIO_WritePin(A_CS0_N_GPIO_Port, A_CS0_N_Pin, GPIO_PIN_SET);

			update_bias_ref(data, SPI);
}
		
	
	void adquire_array_data (uint8_t data[], float32_t channel_1[],float32_t channel_2[],float32_t channel_3[],float32_t channel_4[],float32_t channel_5[],float32_t channel_6[],float32_t channel_7[],float32_t channel_8[], uint8_t gain[], SPI_HandleTypeDef *SPI, UART_HandleTypeDef *huart4)
	{
		int debug = 255;
		// read LENGTH_SAMPLES samples
		
		//adc_wreg(GPIO, 0x2C, SPI);				// Led 1 on, led 2 off
		
		adc_send_command(RDATAC, SPI);
		
		while (HAL_GPIO_ReadPin(A_DRDY_N_GPIO_Port, A_DRDY_N_Pin) == GPIO_PIN_SET){}
		
		HAL_GPIO_WritePin(A_CS0_N_GPIO_Port, A_CS0_N_Pin, GPIO_PIN_RESET);
		
		read_data_frame(data, SPI);
			
		channel_1[0] = byte2float(data[1*3], data[1*3 +1], data[1*3 + 2], gain[0]);
		channel_2[0] = byte2float(data[2*3], data[2*3 +1], data[2*3 + 2], gain[1]);
		channel_3[0] = byte2float(data[3*3], data[3*3 +1], data[3*3 + 2], gain[2]);
		channel_4[0] = byte2float(data[4*3], data[4*3 +1], data[4*3 + 2], gain[3]);
		channel_5[0] = byte2float(data[5*3], data[5*3 +1], data[5*3 + 2], gain[4]);
		channel_6[0] = byte2float(data[6*3], data[6*3 +1], data[6*3 + 2], gain[5]);
		channel_7[0] = byte2float(data[7*3], data[7*3 +1], data[7*3 + 2], gain[6]);
		channel_8[0] = byte2float(data[8*3], data[8*3 +1], data[8*3 + 2], gain[7]);
			
		while (HAL_GPIO_ReadPin(A_DRDY_N_GPIO_Port, A_DRDY_N_Pin) == GPIO_PIN_RESET){}
			
		for (int i = 1; i<LENGTH_SAMPLES; i++)
		{
			while (HAL_GPIO_ReadPin(A_DRDY_N_GPIO_Port, A_DRDY_N_Pin) == GPIO_PIN_SET){}
			
			read_data_frame(data, SPI);

			channel_1[i] = byte2float(data[1*3], data[1*3 +1], data[1*3 + 2], gain[0]);
			channel_2[i] = byte2float(data[2*3], data[2*3 +1], data[2*3 + 2], gain[1]);
			channel_3[i] = byte2float(data[3*3], data[3*3 +1], data[3*3 + 2], gain[2]);
			channel_4[i] = byte2float(data[4*3], data[4*3 +1], data[4*3 + 2], gain[3]);
			channel_5[i] = byte2float(data[5*3], data[5*3 +1], data[5*3 + 2], gain[4]);
			channel_6[i] = byte2float(data[6*3], data[6*3 +1], data[6*3 + 2], gain[5]);
			channel_7[i] = byte2float(data[7*3], data[7*3 +1], data[7*3 + 2], gain[6]);
			channel_8[i] = byte2float(data[8*3], data[8*3 +1], data[8*3 + 2], gain[7]);
				
			while (HAL_GPIO_ReadPin(A_DRDY_N_GPIO_Port, A_DRDY_N_Pin) == GPIO_PIN_RESET){}
		}
		
		HAL_GPIO_WritePin(A_CS0_N_GPIO_Port, A_CS0_N_Pin, GPIO_PIN_SET);
		
		//adc_wreg(GPIO, 0x1C,SPI);				// Led 1 off, led 2 on					
		//Serial_println_N(debug, huart4);
	}
	
		void one_shot (uint8_t data[], SPI_HandleTypeDef *SPI, UART_HandleTypeDef *huart4)
	{
		uint8_t zero = 0x00;
		uint8_t cmd = RDATA;
	// IPIN_MASTER_CS	
		
			//while (HAL_GPIO_ReadPin(A_DRDY_N_GPIO_Port, A_DRDY_N_Pin) == GPIO_PIN_SET){}
		
			HAL_GPIO_WritePin(A_CS0_N_GPIO_Port, A_CS0_N_Pin, GPIO_PIN_RESET);
			
			HAL_SPI_TransmitReceive(SPI, &cmd, &zero,  1, 100);
		
			read_data_frame(data, SPI);
			
			//while (HAL_GPIO_ReadPin(A_DRDY_N_GPIO_Port, A_DRDY_N_Pin) == GPIO_PIN_RESET) {}			
				
			HAL_GPIO_WritePin(A_CS0_N_GPIO_Port, A_CS0_N_Pin, GPIO_PIN_SET);

			update_bias_ref(data, SPI);
}
	
		void one_shot_array (uint8_t data[],float32_t channel_X[], uint8_t channel, uint8_t gain, SPI_HandleTypeDef *SPI, UART_HandleTypeDef *huart4)
	{
		uint8_t zero = 0x00;
		uint8_t cmd = RDATA;
		float32_t anterior = 0x00;
		float32_t pre_anterior = 0x00;
		
	// IPIN_MASTER_CS	
		
			//while (HAL_GPIO_ReadPin(A_DRDY_N_GPIO_Port, A_DRDY_N_Pin) == GPIO_PIN_SET){}
			for(int i = 0; i<2; i++)
		{
				HAL_GPIO_WritePin(A_CS0_N_GPIO_Port, A_CS0_N_Pin, GPIO_PIN_RESET);
					
				HAL_SPI_TransmitReceive(SPI, &cmd, &zero,  1, 100);
				
				//while(HAL_GPIO_ReadPin(A_DRDY_N_GPIO_Port, A_DRDY_N_Pin) == GPIO_PIN_RESET){}
				
				read_data_frame(data, SPI);
			
				channel_X[i] = byte2float(data[channel*3], data[channel*3 +1], data[channel*3 + 2], gain);
				//while (HAL_GPIO_ReadPin(A_DRDY_N_GPIO_Port, A_DRDY_N_Pin) == GPIO_PIN_RESET) {}			
				if (i==0){
					pre_anterior = channel_X[i] = byte2float(data[channel*3], data[channel*3 +1], data[channel*3 + 2], gain);
				}
//				else if (i==1){
//					anterior = channel_X[i] = byte2float(data[channel*3], data[channel*3 +1], data[channel*3 + 2]);
//				}
				HAL_GPIO_WritePin(A_CS0_N_GPIO_Port, A_CS0_N_Pin, GPIO_PIN_SET);
				
				update_bias_ref(data, SPI);
			}
		
		for(int i = 2; i<LENGTH_SAMPLES;)
		{
			if 	(HAL_GPIO_ReadPin(A_DRDY_N_GPIO_Port, A_DRDY_N_Pin) == GPIO_PIN_RESET)
			{
				HAL_GPIO_WritePin(A_CS0_N_GPIO_Port, A_CS0_N_Pin, GPIO_PIN_RESET);
					
				HAL_SPI_TransmitReceive(SPI, &cmd, &zero,  1, 100);
				
				//while(HAL_GPIO_ReadPin(A_DRDY_N_GPIO_Port, A_DRDY_N_Pin) == GPIO_PIN_RESET){}
				
				read_data_frame(data, SPI);
			
				channel_X[i] = byte2float(data[channel*3], data[channel*3 +1], data[channel*3 + 2], gain);
				//while (HAL_GPIO_ReadPin(A_DRDY_N_GPIO_Port, A_DRDY_N_Pin) == GPIO_PIN_RESET) {}			
					
				HAL_GPIO_WritePin(A_CS0_N_GPIO_Port, A_CS0_N_Pin, GPIO_PIN_SET);
				
				update_bias_ref(data, SPI);
				if (channel_X[i] != anterior){
//					if (channel_X[i-1] != pre_anterior){
//					pre_anterior = anterior;
					anterior = channel_X[i];
					i++;
//					}
				}
			}
		}
}
	
uint8_t calcular_ganancia (uint8_t config_channel)
{
	uint8_t ganancia = 12;
	
	switch (config_channel&(0x70))
		{
		case (0x00):
			ganancia = 1;
		break;
		
		case (0x10):
			ganancia = 2;
		break;
		
		case (0x30):
			ganancia = 4;
		break;
		
		case (0x40):
			ganancia = 8;
		break;
		
		case (0x50):
			ganancia = 12;
		break;
		
		case (0x60):
			ganancia = 24;
		break;
		
		default:
			ganancia = 24;
		}
	return ganancia;
}
	
/*****************************END OF FILE*****************************/

\end{lstlisting}

\begin{lstlisting}[label=algoritmo:STM32F4:ADS1299.h,style = STM-code,frame=single,caption=STM32F4:ADS1299.h]
/**
  ******************************************************************************
  * File Name          : ADS1299.hpp
  * Description        : This file contains the common defines of the ADS1299
  ******************************************************************************
  * 
  *
  ******************************************************************************
  */

/* Define to prevent recursive inclusion -------------------------------------*/
#ifndef __ADS1299_H
#define __ADS1299_H
#ifdef __cplusplus
	extern "C" {
#endif
		
		#include "stm32f4xx_hal.h"
		#include "arm_math.h"

		// system commands
#define		WAKEUP 			0x02
#define		STANDBY  		0x04
#define		RESSET  		0x06
#define		START  			0x08
#define		STOP  			0x0A

		// read commands
#define		RDATAC  		0x10
#define		SDATAC  		0x11
#define		RDATA  			0x12

		// register commands
#define		RREG  			0x20
#define		WREG  			0x40
		

		// device settings
#define		ID  				0x00

		// global settings
#define		CONFIG1 	 	0x01
#define		CONFIG2  		0x02
#define		CONFIG3  		0x03
#define		LOFF  			0x04

		// channel specific settings
#define		CHnSET  		0x04
#define		CH1SET  CHnSET + 1
#define		CH2SET  CHnSET + 2
#define		CH3SET  CHnSET + 3
#define		CH4SET  CHnSET + 4
#define		CH5SET  CHnSET + 5
#define		CH6SET  CHnSET + 6
#define		CH7SET  CHnSET + 7
#define		CH8SET  CHnSET + 8
#define		RLD_SENSP  	0x0D
#define		RLD_SENSN  	0x0E
#define		LOFF_SENSP 	0x0F
#define		LOFF_SENSN 	0x10
#define		LOFF_FLIP  	0x11

		// lead off status
#define		LOFF_STATP  0x12
#define		LOFF_STATN  0x13

		// other
#define		GPIO  			0x14
#define		PACE  			0x15
#define		RESP  			0x16
#define		CONFIG4  		0x17
#define		WCT1  			0x18
#define		WCT2  			0x19
	

	// enum CONFIG1_bits {
#define		CONFIG1_reserved  0x90
#define		DAISY_EN  	0x40
#define		CLK_EN  		0x20

#define		DR_250_SPS  0x06
#define		DR_500_SPS  0x05
#define		DR_1_kSPS   0x04
#define		DR_2_kSPS   0x03
#define		DR_4_kSPS  	0x02
#define		DR_8_kSPS  	0x01
#define		DR_16_kSPS  0x00

	// enum CONFIG2_bits {
#define		CONFIG2_reserved  0xC0
#define		INT_CAL  					0x10
#define		CAL_AMP  					0x04
#define		CAL_FREQ_SLOW  		0x00
#define		CAL_FREQ_FAST  		0x01
#define		CAL_FREQ_DC  			0x03


	// enum CONFIG3_bits {
#define		PD_REFBUF		  		0x80
#define		CONFIG3_reserved 	0x60
#define		BIAS_MEAS  				0x10
#define		BIASREF_INT		  	0x08
#define		PD_BIAS  					0x04
#define		BIAS_LOFF_SENS 		0x02
#define		BIAS_STAT  				0x01
	

	// enum LOFF_bits {
#define		COMP_TH2  				0x80
#define		COMP_TH1 		 			0x40
#define		COMP_TH0  				0x20
#define		VLEAD_OFF_EN 		 	0x10
#define		ILEAD_OFF1  			0x08
#define		ILEAD_OFF0  			0x04
#define		FLEAD_OFF1  			0x02
#define		FLEAD_OFF0  			0x01

#define		LOFF_const  			0x00

#define		COMP_TH_95  0x00
#define		COMP_TH_92_5  COMP_TH0
#define		COMP_TH_90  COMP_TH1
#define		COMP_TH_87_5  (COMP_TH1 | COMP_TH0)
#define		COMP_TH_85  COMP_TH2
#define		COMP_TH_80  (COMP_TH2 | COMP_TH0)
#define		COMP_TH_75  (COMP_TH2 | COMP_TH1)
#define		COMP_TH_70  (COMP_TH2 | COMP_TH1 | COMP_TH0)

#define		ILEAD_OFF_6nA  0x00
#define		ILEAD_OFF_12nA  ILEAD_OFF0
#define		ILEAD_OFF_18nA  ILEAD_OFF1
#define		ILEAD_OFF_24nA  (ILEAD_OFF1 | ILEAD_OFF0)

#define		FLEAD_OFF_AC  FLEAD_OFF0
#define		FLEAD_OFF_DC  (FLEAD_OFF1 | FLEAD_OFF0)
	

	// enum CHnSET_bits {
#define		PDn  0x80
#define		PD_n  0x80
#define		GAINn2  0x40
#define		GAINn1  0x20
#define		GAINn0  0x10
#define		SRB2  0x08	// actually ADS1299 specific
#define		MUXn2  0x04
#define		MUXn1  0x02
#define		MUXn0  0x01

#define		CHnSET_const  0x00

#define		GAIN_1X  	0x00
#define		GAIN_2X  	0x10
#define		GAIN_4X  	0x20
#define		GAIN_6X  	0x30
#define		GAIN_8X  	0x40
#define		GAIN_12X  0x50
#define		GAIN_24X  0x60

#define		ELECTRODE_INPUT  0x00
#define		SHORTED  MUXn0
#define		RLD_INPUT  MUXn1
#define		MVDD  (MUXn1 | MUXn0)
#define		TEMP  MUXn2
#define		TEST_SIGNAL  (MUXn2 | MUXn0)
#define		RLD_DRP  (MUXn2 | MUXn1)
#define		RLD_DRN  (MUXn2 | MUXn1 | MUXn0)
	

	// enum CH1SET_bits {
#define		PD_1  0x80
#define		GAIN12  0x40
#define		GAIN11  0x20
#define		GAIN10  0x10
#define		MUX12  0x04
#define		MUX11  0x02
#define		MUX10  0x01

#define		CH1SET_const  0x00
	

	// enum CH2SET_bits {
#define		PD_2  0x80
#define		GAIN22  0x40
#define		GAIN21  0x20
#define		GAIN20  0x10
#define		MUX22  0x04
#define		MUX21  0x02
#define		MUX20  0x01

#define		CH2SET_const  0x00


//	enum CH3SET_bits {
#define		PD_3  0x80
#define		GAIN32  0x40
#define		GAIN31  0x20
#define		GAIN30  0x10
#define		MUX32  0x04
#define		MUX31  0x02
#define		MUX30  0x01

#define		CH3SET_const  0x00


//	enum CH4SET_bits {
#define		PD_4  0x80
#define		GAIN42  0x40
#define		GAIN41  0x20
#define		GAIN40  0x10
#define		MUX42  0x04
#define		MUX41  0x02
#define		MUX40  0x01

#define		CH4SET_const  0x00


//	enum CH5SET_bits {
#define		PD_5  0x80
#define		GAIN52  0x40
#define		GAIN51  0x20
#define		GAIN50  0x10
#define		MUX52  0x04
#define		MUX51  0x02
#define		MUX50  0x01

#define		CH5SET_const  0x00


//	enum CH6SET_bits {
#define		PD_6  0x80
#define		GAIN62  0x40
#define		GAIN61  0x20
#define		GAIN60  0x10
#define		MUX62  0x04
#define		MUX61  0x02
#define		MUX60  0x01

#define		CH6SET_const  0x00


//	enum CH7SET_bits {
#define		PD_7  0x80
#define		GAIN72  0x40
#define		GAIN71  0x20
#define		GAIN70  0x10
#define		MUX72  0x04
#define		MUX71  0x02
#define		MUX70  0x01

#define		CH7SET_const  0x00


//	enum CH8SET_bits {
#define		PD_8  0x80
#define		GAIN82  0x40
#define		GAIN81  0x20
#define		GAIN80  0x10
#define		MUX82  0x04
#define		MUX81  0x02
#define		MUX80  0x01

#define		CH8SET_const  0x00


//	enum RLD_SENSP_bits {
#define		RLD8P  0x80
#define		RLD7P  0x40
#define		RLD6P  0x20
#define		RLD5P  0x10
#define		RLD4P  0x08
#define		RLD3P  0x04
#define		RLD2P  0x02
#define		RLD1P  0x01

#define		RLD_SENSP_const  0x00


//	enum RLD_SENSN_bits {
#define		RLD8N  0x80
#define		RLD7N  0x40
#define		RLD6N  0x20
#define		RLD5N  0x10
#define		RLD4N  0x08
#define		RLD3N  0x04
#define		RLD2N  0x02
#define		RLD1N  0x01

#define		RLD_SENSN_const  0x00


//	enum LOFF_SENSP_bits {
#define		LOFF8P  0x80
#define		LOFF7P  0x40
#define		LOFF6P  0x20
#define		LOFF5P  0x10
#define		LOFF4P  0x08
#define		LOFF3P  0x04
#define		LOFF2P  0x02
#define		LOFF1P  0x01

#define		LOFF_SENSP_const  0x00


//	enum LOFF_SENSN_bits {
#define		LOFF8N  0x80
#define		LOFF7N  0x40
#define		LOFF6N  0x20
#define		LOFF5N  0x10
#define		LOFF4N  0x08
#define		LOFF3N  0x04
#define		LOFF2N  0x02
#define		LOFF1N  0x01

#define		LOFF_SENSN_const  0x00


//	enum LOFF_FLIP_bits {
#define		LOFF_FLIP8  0x80
#define		LOFF_FLIP7  0x40
#define		LOFF_FLIP6  0x20
#define		LOFF_FLIP5  0x10
#define		LOFF_FLIP4  0x08
#define		LOFF_FLIP3  0x04
#define		LOFF_FLIP2  0x02
#define		LOFF_FLIP1  0x01

#define		LOFF_FLIP_const  0x00


//	enum LOFF_STATP_bits {
#define		IN8P_OFF  0x80
#define		IN7P_OFF  0x40
#define		IN6P_OFF  0x20
#define		IN5P_OFF  0x10
#define		IN4P_OFF  0x08
#define		IN3P_OFF  0x04
#define		IN2P_OFF  0x02
#define		IN1P_OFF  0x01

#define		LOFF_STATP_const  0x00


//	enum LOFF_STATN_bits {
#define		IN8N_OFF  0x80
#define		IN7N_OFF  0x40
#define		IN6N_OFF  0x20
#define		IN5N_OFF  0x10
#define		IN4N_OFF  0x08
#define		IN3N_OFF  0x04
#define		IN2N_OFF  0x02
#define		IN1N_OFF  0x01

#define		LOFF_STATN_const  0x00


//	enum GPIO_bits {
#define		GPIOD4  0x80
#define		GPIOD3  0x40
#define		GPIOD2  0x20
#define		GPIOD1  0x10
#define		GPIOC4  0x08
#define		GPIOC3  0x04
#define		GPIOC2  0x02
#define		GPIOC1  0x01

#define		GPIO_const  0x0F


//	enum PACE_bits {
#define		PACEE1  0x10
#define		PACEE0  0x08
#define		PACEO1  0x04
#define		PACEO0  0x02
#define		PD_PACE  0x01

#define		PACE_const  0x00

#define		PACEE_CHAN2  0x00
#define		PACEE_CHAN4  PACEE0
#define		PACEE_CHAN6  PACEE1
#define		PACEE_CHAN8  (PACEE1 | PACEE0)

#define		PACEO_CHAN1  0x00
#define		PACEO_CHAN3  PACEE0
#define		PACEO_CHAN5  PACEE1
#define		PACEO_CHAN7  (PACEE1 | PACEE0)

//	enum RESP_bits {
#define		RESP_DEMOD_EN1  0x80
#define		RESP_MOD_EN1  0x40
#define		RESP_PH2  0x10
#define		RESP_PH1  0x08
#define		RESP_PH0  0x04
#define		RESP_CTRL1  0x02
#define		RESP_CTRL0  0x01

#define		RESP_const  0x20

#define		RESP_PH_22_5  0x00
#define		RESP_PH_45  RESP_PH0
#define		RESP_PH_67_5  RESP_PH1
#define		RESP_PH_90  (RESP_PH1 | RESP_PH0)
#define		RESP_PH_112_5  RESP_PH2
#define		RESP_PH_135  (RESP_PH2 | RESP_PH0)
#define		RESP_PH_157_5  (RESP_PH2 | RESP_PH1)

#define		RESP_NONE  0x00
#define		RESP_EXT  RESP_CTRL0
#define		RESP_INT_SIG_INT  RESP_CTRL1
#define		RESP_INT_SIG_EXT  (RESP_CTRL1 | RESP_CTRL0)


//	enum CONFIG4_bits {
#define		SINGLE_SHOT  		0x08
#define		PD_LOFF_COMP  	0x02

#define		CONFIG4_reserved  0x00

#define		RESP_FREQ_64k_Hz  0x00
#define		RESP_FREQ_32k_Hz  RESP_FREQ0
#define		RESP_FREQ_16k_Hz  RESP_FREQ1
#define		RESP_FREQ_8k_Hz  (RESP_FREQ1 | RESP_FREQ0)
#define		RESP_FREQ_4k_Hz  RESP_FREQ2
#define		RESP_FREQ_2k_Hz  (RESP_FREQ2 | RESP_FREQ0)
#define		RESP_FREQ_1k_Hz  (RESP_FREQ2 | RESP_FREQ1)
#define		RESP_FREQ_500_Hz  (RESP_FREQ2 | RESP_FREQ1 | RESP_FREQ0)


	//enum WCT1_bits {
#define		aVF_CH6  0x80
#define		aVL_CH5  0x40
#define		aVR_CH7  0x20
#define		avR_CH4  0x10
#define		PD_WCTA  0x08
#define		WCTA2  0x04
#define		WCTA1  0x02
#define		WCTA0  0x01

#define		WCT1_const  0x00

#define		WCTA_CH1P  0x00
#define		WCTA_CH1N  WCTA0
#define		WCTA_CH2P  WCTA1
#define		WCTA_CH2N  (WCTA1 | WCTA0)
#define		WCTA_CH3P  WCTA2
#define		WCTA_CH3N  (WCTA2 | WCTA0)
#define		WCTA_CH4P  (WCTA2 | WCTA1)
#define		WCTA_CH4N  (WCTA2 | WCTA1 | WCTA0)


	//enum WCT2_bits {
#define		PD_WCTC  0x80
#define		PD_WCTB  0x40
#define		WCTB2  0x20
#define		WCTB1  0x10
#define		WCTB0  0x08
#define		WCTC2  0x04
#define		WCTC1  0x02
#define		WCTC0  0x01

#define		WCT2_const  0x00

#define		WCTB_CH1P  0x00
#define		WCTB_CH1N  WCTB0
#define		WCTB_CH2P  WCTB1
#define		WCTB_CH2N  (WCTB1 | WCTB0)
#define		WCTB_CH3P  WCTB2
#define		WCTB_CH3N  (WCTB2 | WCTB0)
#define		WCTB_CH4P  (WCTB2 | WCTB1)
#define		WCTB_CH4N  (WCTB2 | WCTB1 | WCTB0)

#define		WCTC_CH1P  0x00
#define		WCTC_CH1N  WCTC0
#define		WCTC_CH2P  WCTC1
#define		WCTC_CH2N  (WCTC1 | WCTC0)
#define		WCTC_CH3P  WCTC2
#define		WCTC_CH3N  (WCTC2 | WCTC0)
#define		WCTC_CH4P  (WCTC2 | WCTC1)
#define		WCTC_CH4N  (WCTC2 | WCTC1 | WCTC0)

	
#define size 27

	// actual size today is 64, but a few extra will not hurt
#define DATA_BUF_SIZE 80

// Tamaño del buffer de la señal a leer
#define DATA_LONG 256
#define CANAL 1

#ifndef LIVE_CHANNELS_NUM
#define LIVE_CHANNELS_NUM 8
#endif
		

void Blinky(SPI_HandleTypeDef *hspi2);
void adc_send_command(uint8_t cmd, SPI_HandleTypeDef *SPI);
uint8_t adc_rreg(uint8_t reg, SPI_HandleTypeDef *SPI);
void adc_wreg(uint8_t reg, uint8_t val, SPI_HandleTypeDef *SPI);
void read_data_frame(uint8_t data[], SPI_HandleTypeDef *SPI);
void update_bias_ref(uint8_t data[], SPI_HandleTypeDef *SPI);
long extrae_un_canal (uint8_t data[], uint8_t canal);
void format_data_frame(uint8_t data[], char *byte_buf);
void print_chip_id(SPI_HandleTypeDef *SPI, UART_HandleTypeDef *huart4);
float32_t byte2float (uint8_t data_23_16, uint8_t data_15_8, uint8_t data_7_0, uint8_t ganancia);
void configADS(uint8_t config[], uint8_t config_channel[], SPI_HandleTypeDef *SPI, UART_HandleTypeDef *huart4);
void adquire_single_data (uint8_t data[], SPI_HandleTypeDef *SPI, UART_HandleTypeDef *huart4);
void adquire_array_data ( 
			uint8_t data[], 
			float32_t channel_1[],
			float32_t channel_2[],
			float32_t channel_3[],
			float32_t channel_4[],
			float32_t channel_5[],
			float32_t channel_6[],
			float32_t channel_7[],
			float32_t channel_8[], 
			uint8_t gain[], SPI_HandleTypeDef *SPI, UART_HandleTypeDef *huart4);
void one_shot (uint8_t data[], SPI_HandleTypeDef *SPI, UART_HandleTypeDef *huart4);
void one_shot_array (uint8_t data[],float32_t channel_X[], uint8_t channel, uint8_t gain, SPI_HandleTypeDef *SPI, UART_HandleTypeDef *huart4);
uint8_t calcular_ganancia (uint8_t config_channel);

uint8_t frame_loff_statp(uint8_t data[]);
uint8_t frame_loff_statn(uint8_t data[]);
uint8_t frame_loff_statp_i(uint8_t data[], int i);
uint8_t frame_loff_statn_i(uint8_t data[], int i);

#ifdef __cplusplus
}
#endif	// __Cplusplus
#endif /*__ADS1299_H */
/*****************************END OF FILE*****************************/

\end{lstlisting}





%AÑADIR DE ANEXO EL BOM

% Fin del documento
\end{document}
