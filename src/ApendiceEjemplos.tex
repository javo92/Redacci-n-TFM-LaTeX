\chapter{Ejemplos de bloques y comandos útiles en LaTeX\label{sec:ejemplos}}
\section{Ejemplo de sección}

%
% Breve guía de comandos útiles para la memoria
%

% Citar una referencia
%La DARPA creó el protocolo de Internet \cite{ipv4sta}.

% Citar un elemento del glosario
Citamos el acrónimo \gls{PCB}.

% Citar un elemento del glosario (primera letra en may´usculas)
\Gls{bitstream} es una secuencia de bits.

% Insertar una imagen con pie de página
\begin{figure}[htp!]
  \centering
  \includegraphics[width=0.75\textwidth,clip=true]{logo_politecnica}
  \caption{Logo de la Universidad Politécnica de madrid.}
  \label{fig:logo_uam}
\end{figure} 

% Referenciar una etiqueta (label)
La figura~\ref{fig:logo_uam} se utiliza en la portada.

% Nueva página
\clearpage

% Añadir código fuente sin líneas
\begin{lstlisting}[label=algoritmo:quicksort,language=C,frame=single,caption=Algoritmo de ordenación Quicksort]
#include <stdio.h>
 
void quick_sort (int *a, int n) {
    int i, j, p, t;
    if (n < 2)
        return;
    p = a[n / 2];
    for (i = 0, j = n - 1;; i++, j--) {
        while (a[i] < p)
            i++;
        while (p < a[j])
            j--;
        if (i >= j)
            break;
        t = a[i];
        a[i] = a[j];
        a[j] = t;
    }
    quick_sort(a, i);
    quick_sort(a + i, n - i);
}
\end{lstlisting}

% Bloque de código inseparable
\begin{code}
#include <stdio.h>
 
void quick_sort (int *a, int n) {
    int i, j, p, t;
    if (n < 2)
        return;
    p = a[n / 2];
    for (i = 0, j = n - 1;; i++, j--) {
        while (a[i] < p)
            i++;
        while (p < a[j])
            j--;
        if (i >= j)
            break;
        t = a[i];
        a[i] = a[j];
        a[j] = t;
    }
    quick_sort(a, i);
    quick_sort(a + i, n - i);
}
\end{code}

% Fórmula dentro de una línea de texto
La ecuación de Euler ($e^{ \pm i\theta } = \cos \theta \pm i\sin \theta$) es citada frecuentemente como un ejemplo de belleza matemática.

% Fórmula independiente
\begin{equation}\label{eq:pythagoras}
a^2 + b^2 = c^2
\end{equation}


\begin{lstlisting}[language=Python, caption=Python example]
import numpy as np
    
def incmatrix(genl1,genl2):
    m = len(genl1)
    n = len(genl2)
    M = None #to become the incidence matrix
    VT = np.zeros((n*m,1), int)  #dummy variable
    
    #compute the bitwise xor matrix
    M1 = bitxormatrix(genl1)
    M2 = np.triu(bitxormatrix(genl2),1) 

    for i in range(m-1):
        for j in range(i+1, m):
            [r,c] = np.where(M2 == M1[i,j])
            for k in range(len(r)):
                VT[(i)*n + r[k]] = 1;
                VT[(i)*n + c[k]] = 1;
                VT[(j)*n + r[k]] = 1;
                VT[(j)*n + c[k]] = 1;
                
                if M is None:
                    M = np.copy(VT)
                else:
                    M = np.concatenate((M, VT), 1)
                
                VT = np.zeros((n*m,1), int)
\end{lstlisting}

\lstlistoflistings

\section{Bill Of Material (BOM)\label{sec:BOM}}